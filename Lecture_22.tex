\input{notes.tex}


\iftoggle{dualscreen}{\setbeameroption{show notes on second screen=right}}{}
\usetikzlibrary{arrows}
\usetikzlibrary{decorations.markings}

\tikzset{
    set arrow inside/.code={\pgfqkeys{/tikz/arrow inside}{#1}},
    set arrow inside={end/.initial=stealth, opt/.initial={scale=2}},
    /pgf/decoration/Mark/.style={
        mark/.expanded=at position #1 with
        {
            \noexpand\arrow[\pgfkeysvalueof{/tikz/arrow inside/opt}]{\pgfkeysvalueof{/tikz/arrow inside/end}}
        }
    },
    arrow inside/.style 2 args={
        set arrow inside={#1},
        postaction={
            decorate,decoration={
                markings,Mark/.list={#2}
            }
        }
    },
}
\newcommand{\cellheight}{0.6\textheight}
\newcommand{\cellwidth}{.33\columnwidth}
\newcommand{\picheight}{0.6\textheight}
\pgfplotsset{compat = newest}
\begin{document}
\section{Lecture 22}
\subsection{Preamble}


\slide[Recall: Stability of fixed points]{
Solutions flow towards or away from fixed points. \subitem{Stable fixed points attract nearby solutions. \item Unstable fixed points repel nearby solutions.}

\ex{} \[ \dd{y}{t} = y \left( 5-y\right) \]\vfill
\twomini[.5]{.5}{.45}{
Two Fixed Points:
\algn{   y^*&=0 &\text{(unstable)}\intertext{and}
y^*&=5&\text{(stable)}}
}{\vfill
\includegraphics[width=\columnwidth]{images/2-2-logistic-mbx.pdf}
}
}


\subsection{Phase Plane Behaviour}

\slide[Qualitative behaviour of solutions: Stability]{
The origin $\vec{0}$ is always a fixed point for the system  $\dd{}{t}\vec{x} = \mathbf{A}(t) \vec{x}$.
\vfill
Q: Do solutions approach or move away from this fixed point?\\
A: It depends on the eigenvalues, $\lambda$, of $\mathbf{A}$.
\vfill
\student{
\itmz{\item If all $\lambda$'s have ${\rm Re}(\lambda)<0$, solutions approach $\vec{0}$ as $t\to\infty$.\vfill\subitem{\alert{Stable fixed point}}\vfill
\item If any $\lambda$'s have ${\rm Re}(\lambda)>0$, solutions move away from $\vec{0}$  as $t\to\infty$.\vfill\subitem{\alert{Unstable fixed point}}\vfill
\item If a complex conjugate $\lambda$ pairs exists, solutions  exhibit oscillations around $\vec{0}$.\vfill
\subitem{\alert{Spiral fixed point}}} \vfill
}


}
\slide[Classifying Fixed Points]{
The origin $\vec{0}$ is always a fixed point for the system  $\dd{}{t}\vec{x} = \mathbf{A}(t) \vec{x}$.
\vfill

\student{
 If all eigevalues are real and ...\vfill
\itmz{\item they have the same sign, the fixed point is called a \alert{node}\vfill \subitem{All $\lambda\text{'s}<0 \quad \Rightarrow$ \alert{stable node} (sink) \vfill \item All $\lambda\text{'s}>0 \quad \Rightarrow$ \alert{unstable node} (source)}\vfill
\item some $\lambda$'s have opposite signs, the fixed point is called a \alert{saddle}
}
}
}



\slide[Classifying Fixed Points]{
The origin $\vec{0}$ is always a fixed point for the system  $\dd{}{t}\vec{x} = \mathbf{A}(t) \vec{x}$.\vfill
\student{
 If a pair of eigenvalues are complex conjugates, $\lambda_{1,2}=r \pm i \omega$, then  the fixed point is called a \alert{spiral} (or focus).\vfill
\itmz{\item For a $2\times2$ matrix, with \vfill\subitem{$r<0\quad \Rightarrow$ \alert{stable spiral} (spiral sink) \vfill \item $r>0\quad \Rightarrow$ \alert{unstable spiral} (spiral source)\vfill
\item $r=0\quad \Rightarrow$ \alert{neutral spiral} (spiral center)
}\vfill
}
}
}

\slide[2D Vector Fields]{
The ODE \[\dd{}{t}\vec{x} = \mathbf{A}(t)\vec{x}\] gives us a derivative for each point in $\vec{x}\in\mathbb{R}^n$\vfill
\student{
Restricting ourselves to $\mathbb{R}^2$ with constant $\mathbf{A}$, we can draw an arrow parallel to the derivative at many points in the $(x,y)$-plane.
\vfill
Then we can visualize approximate solution flows in the plane.}
}







\slide{
\hfill$\uline{\lambda_{1,2}\in\mathbb{R}}$ with $\left| \textcolor{Plum}{\lambda_2} \right|>\left| \textcolor{YellowOrange}{\lambda_1} \right|$
\vfill

\student{Each eigenvalue/eigenvector is associated with an \alert{eigendirection}\vfill\subitem{Solutions flow as straight lines along the eigendirection}}

\vfill

\twomini[.6]{.45}{.45}{

\centering
Unstable Node:
$0<\textcolor{YellowOrange}{\lambda_1}<\textcolor{Plum}{\lambda_2}$
\vfill

\centerline{
\begin{tikzpicture}
\begin{axis}[
	xtick={-2,0,2},
	ytick={-3,0,3},
    xmin = -4, xmax = 4,
    ymin = -5, ymax = 5,
    zmin = 0, zmax = 1,
    width=\columnwidth*1.15,
    height=\picheight,
    view = {0}{90},
    xlabel={},
    ylabel={}
]
\draw[ultra thick , Plum, opacity=0.3]  (-6*0.618034,-6*1) -- (6*0.618034,1*6) ;
\draw[ultra thick ,  YellowOrange, opacity=0.3]  (-6*-1.61803,-6*1) -- (6*-1.61803,1*6)  ;
    \addplot3[
        quiver = {
            u = {(x+y)/sqrt((x+2*y)^2+(x+y)^2)},
            v = {(x+2*y)/sqrt((x+2*y)^2+(x+y)^2)},
            scale arrows = 0.4,
                every arrow/.append style={%
                    line width=.1+\pgfplotspointmetatransformed/2500,
                    -{Latex[length=0pt 5,width=0pt 3]}
                }
        },
        -stealth,
        domain = -4:4,
        domain y = -5:5,
        samples=20
    ] {0};
    



%\draw[] plot[domain=0:1, samples=10] (-2*exp(-2*\x) + 3*exp(3*\x), exp(-2\x)+exp(3\x));
\end{axis}



\end{tikzpicture}
}
}{

\centering
Stable Node:
$\textcolor{Plum}{\lambda_2}<\textcolor{YellowOrange}{\lambda_1}<0$
\vfill


\centerline{
\begin{tikzpicture}
\begin{axis}[
	xtick={-2,0,2},
	ytick={-3,0,3},
    xmin = -4, xmax = 4,
    ymin = -5, ymax = 5,
    zmin = 0, zmax = 1,
    width=\columnwidth*1.15,
    height=\picheight,
    view = {0}{90},
    xlabel={},
    ylabel={},
]

\draw[ultra thick , Plum, opacity=0.3]  (-6*0.618034,-6*1) -- (6*0.618034,1*6) ;
\draw[ultra thick ,  YellowOrange, opacity=0.3]  (-6*-1.61803,-6*1) -- (6*-1.61803,1*6)  ;

    \addplot3[
        quiver = {
            u = {(-x+-y)/sqrt((x+2*y)^2+(x+y)^2)},
            v = {(-x+-2*y)/sqrt((x+2*y)^2+(x+y)^2)},
            scale arrows = 0.4,
                every arrow/.append style={%
                    line width=.1+\pgfplotspointmetatransformed/2500,
                    -{Latex[length=0pt 5,width=0pt 3]}
                }
        },
        -stealth,
        domain = -4:4,
        domain y = -5:5,
        samples=20
    ] {0};



%\draw[] plot[domain=0:1, samples=10] (-2*exp(-2*\x) + 3*exp(3*\x), exp(-2\x)+exp(3\x));
\end{axis}



\end{tikzpicture}}
}

}


\slide{Special Case: Zero Eigenvalue\vfill
\begin{minipage}{.47\textwidth}
\centering Stable \\~\\
\begin{tikzpicture}[xscale=0.7, yscale=.7]
\begin{axis}[
    xmin = -2.5, xmax = 2.5,
    ymin = -2.5, ymax = 2.5,
    zmin = 0, zmax = 1,
    axis equal image,
    view = {0}{90},
    xlabel={},
    ylabel={},
]
    \addplot3[
        quiver = {
            u = {-(x+2*y)/sqrt((2*x+4*y)^2+(x+2*y)^2)},
            v = {-(2*x+4*y)/sqrt((2*x+4*y)^2+(x+2*y)^2)},
            scale arrows = 0.2,
                every arrow/.append style={%
                    line width=.1+\pgfplotspointmetatransformed/2500,
                    -{Latex[length=0pt 5,width=0pt 3]}
                }
        },
        -stealth,
        domain = -2.5:2.5,
        domain y = -2.5:2.5,
	samples=20
    ] {0};
\draw[->, thick, black] (0,0) -- (1, 2) node[fill=white, fill opacity=0.8, yshift=-1.4em, xshift=2em]{($\lambda=-5$)};
\draw[->, thick, black] (0,0)--(2,-1)  node[fill=white, fill opacity=0.8, yshift=-.7em, xshift=-2.5em]{($\lambda=0$)};
\draw[-, dashed, black] (-4, 2) -- (4,-2) ;
\end{axis}


\end{tikzpicture}
\centering
\end{minipage}\hfill
\begin{minipage}{.47\textwidth}
\centering Unstable \\~\\
\begin{tikzpicture}[xscale=0.7, yscale=.7]
\begin{axis}[
    xmin = -2.5, xmax = 2.5,
    ymin = -2.5, ymax = 2.5,
    zmin = 0, zmax = 1,
    axis equal image,
    view = {0}{90},
    xlabel={},
    ylabel={},
]
    \addplot3[
        quiver = {
            u = {(x+2*y)/sqrt((2*x+4*y)^2+(x+2*y)^2)},
            v = {(2*x+4*y)/sqrt((2*x+4*y)^2+(x+2*y)^2)},
            scale arrows = 0.2,
                every arrow/.append style={%
                    line width=.1+\pgfplotspointmetatransformed/2500,
                    -{Latex[length=0pt 5,width=0pt 3]}
                }
        },
        -stealth,
        domain = -2.5:2.5,
        domain y = -2.5:2.5,
	samples=20
    ] {0};
\draw[->, thick, black] (0,0) -- (1, 2) node[fill=white, fill opacity=0.8, yshift=-1.4em, xshift=1.8em]{($\lambda=5$)};
\draw[->, thick, black] (0,0)--(2,-1)  node[fill=white, fill opacity=0.8, yshift=-.7em, xshift=-2.5em]{($\lambda=0$)};
\draw[-, dashed, black] (-4, 2) -- (4,-2) ;
\end{axis}


\end{tikzpicture}

\end{minipage}
\vfill
\centerline{Line of fixed-points (non-isolated fixed points).}
}

\slide{

\hfill$\uline{\lambda_{1,2}\in\mathbb{R}}$ with $\left| \textcolor{YellowOrange}{\lambda_1} \right| < \left| \textcolor{Plum}{\lambda_2} \right|$
\vfill

Saddle: $\textcolor{YellowOrange}{\lambda_1}<0<\textcolor{Plum}{\lambda_2}$
\vfill

\centerline{
\begin{tikzpicture}
\begin{axis}[
	xtick={6,-3,0,3, 6},
	ytick={-3,0,3},
    zmin = 0, zmax = 1,
    width=0.8*\columnwidth,
    height=6cm,
    view = {0}{90},
    xlabel={},
    ylabel={},
]
    \addplot3[
        quiver = {
            u = {(x+2*y)/sqrt((3*x+y)^2+(x+2*y)^2)},
            v = {(3*x+y)/sqrt((3*x+y)^2+(x+2*y)^2)},
            scale arrows = 0.3,
                every arrow/.append style={%
                    line width=.1+\pgfplotspointmetatransformed/3000,
                    -{Latex[length=0pt 5,width=0pt 3]}
                }
        },
        -stealth,
        domain = -7:7,
        domain y = -5:5,
        samples=40
    ] {0};
\draw[ultra thick , Plum, opacity=0.4]  ({-9*0.816497},{-9*1}) -- ({9*0.816497},{1*9}) ;
\draw[ultra thick ,  YellowOrange, opacity=0.4]  (-9*-0.816497,-9*1) -- (9*-0.816497,1*9)  ;


%\draw[] plot[domain=0:1, samples=10] (-2*exp(-2*\x) + 3*exp(3*\x), exp(-2\x)+exp(3\x));
\end{axis}



\end{tikzpicture}}\vfill

\student{
\subitem{Eigendirections with ${\rm Re}(\lambda)>0$ are repelling \vfill \item Eigendirections with  ${\rm Re}(\lambda)<0$ are attracting \vfill}
}

}



\slide{Repeated Eigenvalue (\alert{Degenerate Node})\hfill$\uline{\lambda_{1}=\lambda_2\in\mathbb{R}}$
\vfill
\centerline{
\begin{tikzpicture}
\begin{axis}[
    xmin = -2.5, xmax = 2.5,
    ymin = -2.5, ymax = 2.5,
    zmin = 0, zmax = 1,
    width=\columnwidth*0.65,
    height=1.2*\picheight,
    view = {0}{90},
    xlabel={},
    ylabel={},
]
\draw[ultra thick , Plum, opacity=0.3]  (-6,6) -- (6,-6) ;
    \addplot3[
        quiver = {
            u = {(x-y)/sqrt((x+3*y)^2+(x-y)^2)},
            v = {(x+3*y)/sqrt((x+3*y)^2+(x-y)^2)},
            scale arrows = 0.2,
                every arrow/.append style={%
                    line width=.1+\pgfplotspointmetatransformed/2500,
                    -{Latex[length=0pt 5,width=0pt 3]}
                }
        },
        -stealth,
        domain = -2.5:2.5,
        domain y = -2.5:2.5,
	samples=20
    ] {0};

%\draw[->, thick, black] (0,0) -- (1,-1) node[fill=white, fill opacity=0.8, yshift=-1em]{($\lambda=2$)};
%\draw[->, thick, black] (0,0) -- (-1,0)  node[inner sep=0,outer sep=0, fill=white, fill opacity=0.8, yshift=1.2em, align=center]{\small generalized eigenvector \\ \small ($\lambda=2$)};
%\draw[->, thick, \normcolor] (0,0) -- (-1,-1)  ;
%\draw[->, dashed, thick, black] (-1,0) -- (-2,0)  ;
%\draw[->, dashed, thick, black] (-2,0) -- (-1,-1)  ;
\end{axis}


\end{tikzpicture}}\vfill
Far from the origin, solutions align with the \textcolor{Plum}{eigendirection}. Near the origin, they can rotate.


}


\slide[Spiral Fixed-Points ($\lambda_{1,2}=r\pm i\omega$)] {
\vfill
\begin{minipage}[b][0.7\textheight][s]{.49\columnwidth} 
\vspace{-1em}
\centering
$r>0$
\vfill

\hspace{-1em}
\begin{tikzpicture}
\begin{axis}[
	ticks=none,
    xmin = -4, xmax = 4,
    ymin = -5, ymax = 5,
    zmin = 0, zmax = 1,
    width=\columnwidth,
    height=0.65\textheight,
    view = {0}{90},
    xlabel={}, hide x axis,
    ylabel={},hide y axis,
]
    \addplot3[
        quiver = {
            u = {(x+2*y)/sqrt((x-2*y)^2+(3*x+y)^2)},
            v = {(-3*x+y)/sqrt((x-2*y)^2+(3*x+y)^2)},
            scale arrows = 0.4,
                every arrow/.append style={%
                    line width=.1+\pgfplotspointmetatransformed/2000,
                    -{Latex[length=0pt 5,width=0pt 3]}
                }
        },
        -stealth,
        domain = -4:4,
        domain y = -5:5,
        samples=16
    ] {0};



%\draw[] plot[domain=0:1, samples=10] (-2*exp(-2*\x) + 3*exp(3*\x), exp(-2\x)+exp(3\x));
\end{axis}



\end{tikzpicture}

\vfill
\student{
Unstable spiral}
\end{minipage}
\begin{minipage}[b][0.7\textheight][s]{.49\columnwidth} 
\vspace{-1em}
\centering
$r<0$
\vfill

\hspace{-1em}
\begin{tikzpicture}
\begin{axis}[
	ticks=none,
    xmin = -4, xmax = 4,
    ymin = -5, ymax = 5,
    zmin = 0, zmax = 1,
    width=\columnwidth,
    height=0.65\textheight,
    view = {0}{90},
    xlabel={}, hide x axis,
    ylabel={},hide y axis,
]
    \addplot3[
        quiver = {
            u = {-(x+2*y)/sqrt((x-2*y)^2+(3*x+y)^2)},
            v = {-(-3*x+y)/sqrt((x-2*y)^2+(3*x+y)^2)},
            scale arrows = 0.4,
                every arrow/.append style={%
                    line width=.1+\pgfplotspointmetatransformed/2000,
                    -{Latex[length=0pt 5,width=0pt 3]}
                }
        },
        -stealth,
        domain = -4:4,
        domain y = -5:5,
        samples=16
    ] {0};



%\draw[] plot[domain=0:1, samples=10] (-2*exp(-2*\x) + 3*exp(3*\x), exp(-2\x)+exp(3\x));
\end{axis}



\end{tikzpicture}

\vfill
\student{
Stable spiral}
\end{minipage} \vfill
\student{We will see how the eigenvectors dictate the direction of rotation shortly.}
\vfill

}


\slide[Center Fixed-Point ($\lambda_{1,2}= \pm i\omega$)] {
\vfill

\vspace{-1em}
\centering
\vfill

\hspace{-1em}
\begin{tikzpicture}
\begin{axis}[
	ticks=none,
    xmin = -4, xmax = 4,
    ymin = -5, ymax = 5,
    zmin = 0, zmax = 1,
    width=.8\columnwidth,
    height=0.85\textheight,
    view = {0}{90},
    xlabel={}, hide x axis,
    ylabel={},hide y axis,
]
    \addplot3[
        quiver = {
            u = {(-2*y)/sqrt((2*y)^2+(2*x)^2)},
            v = {(2*x)/sqrt((2*y)^2+(2*x)^2)},
            scale arrows = 0.4,
                every arrow/.append style={%
                    line width=.1+\pgfplotspointmetatransformed/2000,
                    -{Latex[length=0pt 5,width=0pt 3]}
                }
        },
        -stealth,
        domain = -4:4,
        domain y = -5:5,
        samples=24
    ] {0};



%\draw[] plot[domain=0:1, samples=10] (-2*exp(-2*\x) + 3*exp(3*\x), exp(-2\x)+exp(3\x));
\end{axis}



\end{tikzpicture}

\vfill
\student{The origin has neutral stability. \\ Solutions travel around the origin indefinitely.}
}




\slide[Sketching Solutions in the Plane ]{
For a $2\times2$ system  $\dd{}{t}\vec{x} = \mathbf{A} \vec{x}$ with real distinct eigenvalues\vfill
\enum{
\item Find the eigenvalues/vectors of the matrix $\mathbf{A}$.\vfill
\item Draw the eigendirections (eigenvectors) of the system.\vfill
\subitem{Eigendirections with  $\lambda>0$ are repelling \vfill \item Eigendirections with $\lambda>0$ are attracting \vfill
}

\item Draw a few sample solution flows. \subitem{As solutions get closer to an eigendirection, they align themselves with that direction.}
}

}



\slide[  Sketch solutions to{ \small\hfill$\larray{ \dd{x}{t} = -3x - 2y \\ \dd{y}{t} = -2x - 6y}$ }\hfill \large in the phase-plane]{\vspace{-.75em}
\small$\vec{x}(t) = c_1\mat{c}{-2\\1}e^{-2t} + c_2 \mat{c}{1\\2}e^{-7t}$\normalsize\vspace{-.75em}
\centerline{\tikzplot[\foreach \i in {-5,-4,-3, -2, -1,1,2, 3,4,5} {\xcoord{\i}{}} \foreach \i in {-3, -2, -1,1,2, 3} {\ycoord{\i}{}}]{6}{6}{3.2}{3.2}{x}{y}{
\student
{

\draw[black ] plot[domain=0:1, smooth] ({-2*exp(-2*\x)+1*exp(-7*\x)},{exp(-2*\x)+2*exp(-7*\x)})  [arrow inside={}{0.5}];
\draw[black] plot[domain=0:2, smooth] ({-2*exp(-2*\x)-1*exp(-7*\x)},{exp(-2*\x)-2*exp(-7*\x)}) [arrow inside={}{0.5}];
\draw[black] plot[domain=0:2, smooth] ({2*exp(-2*\x)-1*exp(-7*\x)},{-exp(-2*\x)-2*exp(-7*\x)}) [arrow inside={}{0.5}];
\draw[black] plot[domain=0:2, smooth] ({2*exp(-2*\x)+1*exp(-7*\x)},{-exp(-2*\x)+2*exp(-7*\x)}) [arrow inside={}{0.5}];

\draw[black] plot[domain=0:2, smooth] ({3*exp(-2*\x)+1*exp(-7*\x)},{-1.5*exp(-2*\x)+2*exp(-7*\x)}) [arrow inside={}{0.5}];
\draw[black] plot[domain=0:2, smooth] ({-3*exp(-2*\x)+1*exp(-7*\x)},{1.5*exp(-2*\x)+2*exp(-7*\x)}) [arrow inside={}{0.5}];
\draw[black] plot[domain=0:2, smooth] ({-3*exp(-2*\x)-1*exp(-7*\x)},{1.5*exp(-2*\x)-2*exp(-7*\x)}) [arrow inside={}{0.5}];
\draw[black] plot[domain=0:2, smooth] ({3*exp(-2*\x)-1*exp(-7*\x)},{-1.5*exp(-2*\x)-2*exp(-7*\x)}) [arrow inside={}{0.5}];

\draw[thick, ->, red] (0,0)--(1,2) node[xshift=1em]{$\vec{v}_2$}; 
\draw[dashed] (-2,-4)  -- (0,0)  [arrow inside={end=stealth,opt={scale=2}}{0.3,0.7}] ; 
\draw[dashed] (2,4)  node[xshift=1.35cm, yshift=-1.5cm, align=center]{Dominant\\ stable eigendirection \\ ($\lambda_2=-7$)} -- (0,0) [arrow inside={}{0.3,0.7}] ; 

\draw[ thick,->, red] (0,0)--(-2,1)node[xshift=-1em,yshift=-.25em]{$\vec{v}_1$};
\draw[dashed] (-5,2.5)--(0,0) [arrow inside={}{0.25,0.5,0.75}]; 
\draw[dashed](5,-2.5) node[xshift=-.75cm, yshift=1.35cm,align=center]{stable eigendirection\\ ($\lambda_1=-2$)} --  (0,0)[arrow inside={}{0.25,0.5,0.75}] ; 

 }
}}
}



\slide[  Sketch solutions to{ \small\hfill$\larray{ \dd{x}{t} = - 2y \\ \dd{y}{t} = -2x - 3y}$ }\hfill \large in the phase-plane]{\vspace{-.75em}
\small$\vec{x}(t) = c_1\mat{c}{-2\\1}e^{t}+c_2\mat{c}{1\\2}e^{-4t}$\normalsize\vspace{-.75em}
\centerline{\tikzplot[\foreach \i in {-5,-4,-3, -2, -1,1,2, 3,4,5} {\xcoord{\i}{}} \foreach \i in {-3, -2, -1,1,2, 3} {\ycoord{\i}{}}]{6}{6}{3.2}{3.2}{x}{y}{
\draw[thick, ->, red] (0,0)--(1,2) node[xshift=1em]{$\vec{v}_2$}; 
\draw[ thick,->, red] (0,0)--(-2,1)node[xshift=-1em,yshift=-.25em]{$\vec{v}_1$};
\student{

\draw[black ] plot[domain=0:1.9, smooth] ({-.75*exp(\x)+1.5*exp(-4*\x)},{.375*exp(\x)+3*exp(-4*\x)})  [arrow inside={}{.25}];
\draw[black ] plot[domain=0:1.9, smooth] ({.75*exp(\x)+1.5*exp(-4*\x)},{-.375*exp(\x)+3*exp(-4*\x)})  [arrow inside={}{.25}];
\draw[black ] plot[domain=0:1.9, smooth] ({.75*exp(\x)-1.5*exp(-4*\x)},{-.375*exp(\x)-3*exp(-4*\x)})  [arrow inside={}{.25}];
\draw[black ] plot[domain=0:1.9, smooth] ({-.75*exp(\x)-1.5*exp(-4*\x)},{.375*exp(\x)-3*exp(-4*\x)})  [arrow inside={}{.25}];

\draw[black ] plot[domain=0:1.7, smooth] ({-1.5*exp(\x)+1.5*exp(-4*\x)},{.75*exp(\x)+3*exp(-4*\x)})  [arrow inside={}{.25}];
\draw[black ] plot[domain=0:1.7, smooth] ({1.5*exp(\x)+1.5*exp(-4*\x)},{-.75*exp(\x)+3*exp(-4*\x)})  [arrow inside={}{.25}];
\draw[black ] plot[domain=0:1.7, smooth] ({-1.5*exp(\x)-1.5*exp(-4*\x)},{.75*exp(\x)-3*exp(-4*\x)})  [arrow inside={}{.25}];
\draw[black ] plot[domain=0:1.7, smooth] ({1.5*exp(\x)-1.5*exp(-4*\x)},{-.75*exp(\x)-3*exp(-4*\x)})  [arrow inside={}{.25}];


\draw[dashed] (-2,-4)  -- (0,0)  [arrow inside={}{0.25,.5,0.75}]; 

\draw[dashed] (2,4)  node[xshift=1.35cm, yshift=-1.25cm, align=center]{stable eigendirection \\ ($\lambda_2=-4$)} -- (0,0)  [arrow inside={}{0.25,.5,0.75}]; 


\draw[dashed] (0,0) -- (-5,2.5) [arrow inside={}{0.25,.5,0.75}]; 
\draw[dashed] (0,0) -- (5,-2.5) node[xshift=-2.15cm, yshift=-.25cm, align=center]{unstable eigendirection\\ ($\lambda_1=1$)}  [arrow inside={}{0.25,.5,0.75}]; 
 }
}}
}

\slide[Sketch the solution behaviours for \hfill \small  $\larray{ \dd{x}{t} = r x + -2y \\ \dd{y}{t} = 2x  + r y}$]{
\scriptsize
$  \vec{x}_1= e^{r t}\left( \mat{c}{1\\0}\cos(2t) - \mat{c}{0\\-1}  \sin(2t) \right) \qquad
\vec{x}_2= e^{r t}\left(\mat{c}{1\\0}\sin(2t) + \mat{c}{0\\-1}\cos(2t)   \right) $ \normalsize\vfill
\centerline{\tikzplot[\foreach \i in {-5,-4,-3, -2, -1,1,2, 3,4,5} {\xcoord{\i}{}} \foreach \i in { -2, -1,1,2} {\ycoord{\i}{}}]{6}{6}{3}{2.5}{x}{y}{
\student{
\draw  (-4.5,1.5) node[align=left]{$\vec{x}_1(0)=\colvec{1\\0}$\\$\vec{x}_1(\pi/4)=\colvec{0\\1}$};
\draw[Cyan,domain=0:800, ultra thick, smooth, samples=50] plot ({2*exp(-\x/180)*(cos(\x))},{2*exp(-\x/180)*(sin(\x))})  [arrow inside={}{0.65}];
\draw[Cyan] (-1,1.05) node{$r<0$};
\draw[domain=0:360, ultra thick, smooth, samples=50] plot ({2*(cos(\x))},{2*(sin(\x))})  [arrow inside={}{0.65}];
\draw[] (1.8, -1.7) node{$r=0$};
\draw[Plum,domain=0:400, ultra thick, smooth, samples=50] plot ({2*exp(\x/700)*(cos(\x))},{2*exp(\x/700)*(sin(\x))})  [arrow inside={}{0.8}];
\draw[Plum] (3.8,1.5) node{$r>0$};
}
}}
}


\slide[Qualitative behaviour of solutions: Chirality]{\vspace{-.5em}
With $\lambda_{1,2}=r\pm i\omega$ and $\vec{v}_{1,2}=\vec{a}\pm i\vec{b}$ we get\vfill
$  \vec{x}_1= e^{r t}\left( \cos(\omega t) \vec{a} -  \sin(\omega t) \vec{b} \right)$\hfill and \hfill $\vec{x}_2= e^{r t}\left(\sin(\omega t) \vec{a} + \cos(\omega t) \vec{b}  \right) $
\vfill
Evaluate either eigensolution at $t=0$ and $\omega t = \pi/2$, the path joining those two points tells you the direction of rotation.
\vfill
Two possibilities:
\vfill
\twomini[.4]{.5}{.5}{

Clockwise (right-handed)


\tikzplot{2}{2}{1.2}{1.35}{x}{y}{
\draw[red, ultra thick, ->] (0,0)--(1.15,1.15) node[right] {$\vec{a}$};
\draw[Cyan, ultra thick, ->] (0,0)--(-1.15,1.15) node[left] {$\vec{b}$};
\draw[Cyan, ultra thick, dashed, ->] (0,0)--(1.15,-1.15) node[left,pos=0.75, xshift=-.15cm] {$-\vec{b}$};
\student{
\draw[domain=0:90, thick] plot ({.7*exp(-\x/250)*(cos(\x)+sin(\x))},{.7*exp(-\x/250)*(cos(\x)-sin(\x))})  [arrow inside={}{0.65}];
\draw[domain=90:400, dashed] plot ({.7*exp(-\x/250)*(cos(\x)+sin(\x))},{.7*exp(-\x/250)*(cos(\x)-sin(\x))})  [arrow inside={}{0.65}];
\draw[] (1.5,.55) node[]{$\vec{x}_1(0)$};
\draw[domain=0:90, thick] plot ({.7*exp(-\x/250)*(-cos(\x)+sin(\x))},{.7*exp(-\x/250)*(cos(\x)+sin(\x))})  [arrow inside={}{0.65}];
\draw[] (-1.3,.55) node[]{$\vec{x}_2(0)$};
}
}

}{

Counter-clockwise (left-handed)


\tikzplot{2}{2}{1.2}{1.35}{x}{y}{
\draw[red, ultra thick, ->] (0,0)--(-1.15,1.15) node[left] {$\vec{a}$};
\draw[Cyan, ultra thick, ->] (0,0)--(1.15,1.15) node[right] {$\vec{b}$};
\draw[Cyan, ultra thick, dashed, ->] (0,0)--(-1.15,-1.15) node[right,pos=0.75]{$-\vec{b}$};

\student{
\draw[domain=0:90, thick] plot ({.3*exp(\x/180)*(-cos(\x)-sin(\x))},{.3*exp(\x/180)*(cos(\x)-sin(\x))})  [arrow inside={}{0.65}];
\draw[domain=90:275, dashed] plot ({.3*exp(\x/180)*(-cos(\x)-sin(\x))},{.3*exp(\x/180)*(cos(\x)-sin(\x))})  [arrow inside={}{0.65}];
\draw[] (-1,.4) node[]{$\vec{x}_1(0)$};
\draw[domain=0:90, thick] plot ({.6*exp(\x/250)*(cos(\x)-sin(\x))},{.6*exp(\x/250)*(cos(\x)+sin(\x))})  [arrow inside={}{0.65}];
\draw[] (1,.35) node[]{$\vec{x}_2(0)$};
}

}
}
}



\slide[  Sketch solutions to{ \small\hfill$\larray{ \dd{x}{t} = x - y \\ \dd{y}{t} = x + 3y}$ }\hfill \large in the phase-plane]{\vspace{-.75em}
\small$\quad \vec{x}=c_1e^{2t}\colvec{1\\-1}  + c_2e^{2t} \left( \colvec{1\\-2}+t\colvec{1\\-1} \right) $\normalsize\vspace{-.75em}
\centerline{\tikzplot[\foreach \i in {-5,-4,-3, -2, -1,1,2, 3,4,5} {\xcoord{\i}{}} \foreach \i in {-3, -2, -1,1,2, 3} {\ycoord{\i}{}}]{6}{6}{3.2}{3.2}{x}{y}{

\student{
\draw[thick, ->, red] (0,0)--(1,-2) node[left, xshift=-.3em, align=center]{$\vec{w}$}; 
\draw[ thick,->, red] (0,0)--(1,-1)node[right, xshift=.2em,yshift=.25em]{$\vec{v}$};
%
%\draw[black ] plot[domain=0:1.9, smooth] ({0.03*exp(2*\x) - 0.02*exp(2*\x)*(1 + \x)},{-0.03*exp(2*\x) - 0.02*exp(2*\x)*(-2 - \x)})  [arrow inside={}{.25}];
%\draw[black ] plot[domain=0:1.9, smooth] ({.75*exp(\x)+1.5*exp(-4*\x)},{-.375*exp(\x)+3*exp(-4*\x)})  [arrow inside={}{.25}];
%\draw[black ] plot[domain=0:1.9, smooth] ({.75*exp(\x)-1.5*exp(-4*\x)},{-.375*exp(\x)-3*exp(-4*\x)})  [arrow inside={}{.25}];
%\draw[black ] plot[domain=0:1.9, smooth] ({-.75*exp(\x)-1.5*exp(-4*\x)},{.375*exp(\x)-3*exp(-4*\x)})  [arrow inside={}{.25}];
%
%\draw[black ] plot[domain=0:1.7, smooth] ({-1.5*exp(\x)+1.5*exp(-4*\x)},{.75*exp(\x)+3*exp(-4*\x)})  [arrow inside={}{.25}];
%\draw[black ] plot[domain=0:1.7, smooth] ({1.5*exp(\x)+1.5*exp(-4*\x)},{-.75*exp(\x)+3*exp(-4*\x)})  [arrow inside={}{.25}];
%\draw[black ] plot[domain=0:0.2, smooth] ({0.00003*exp(2*\x) - 0.00002*exp(2*\x)*(1 + \x)},{-0.00003*exp(2*\x) - 0.00002*exp(2*\x)*(-2 - \x)})  [arrow inside={}{.25}];
%\draw[black ] plot[domain=0:1.7, smooth] ({1.5*exp(\x)-1.5*exp(-4*\x)},{-.75*exp(\x)-3*exp(-4*\x)})  [arrow inside={}{.25}];
\draw[black ] plot[domain=-2:1, smooth] ({-2*exp(2*\x) + (3*exp(2*\x)*(1 + \x))/2},{ 2*exp(2*\x) + (3*exp(2*\x)*(-2 - \x))/2})  [arrow inside={}{.08}];
\draw[black ] plot[domain=-2:1, smooth] ({-exp(2*\x)},{exp(2*\x)})  [arrow inside={}{.25}];

\draw[black ] plot[domain=-2:1, smooth] ({2*exp(2*\x) - (3*exp(2*\x)*(1 + \x))/2},{ -2*exp(2*\x) - (3*exp(2*\x)*(-2 - \x))/2})  [arrow inside={}{.08}];
\draw[black ] plot[domain=-2:1, smooth] ({exp(2*\x)},{-exp(2*\x)})  [arrow inside={}{.25}];

\draw[dashed, thick] (0,0)  -- (-6,0)  []; 

\draw (3,2) node[align=center]{\uline{To Find Rotation Direction:}\\~\\ Draw a line from the tip of $\vec{w}$\\  towards the eigendirection\\ aligned with  $\vec{v}$};


\draw[dashed] (0,0) -- (-3,3) [arrow inside={}{0.25,.5,0.75}]; 
\draw[dashed] (0,0) -- (3,-3) node[xshift=1.15cm, yshift=.75cm, align=center]{unstable eigendirection\\ ($\lambda=2$)}  [arrow inside={}{0.25,.5,0.75}]; 
 }
}}
}



\end{document}