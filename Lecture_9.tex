\input{notes.tex}


\iftoggle{dualscreen}{\setbeameroption{show notes on second screen=right}}{}

\begin{document}
\section{Lecture 9}
\subsection{Introduction}
\settoggle{covered}{true}
\slide[Recall: Constant Coefficient Homogeneous 2$^{nd}$ Order ODE]{\vspace{-1em}
\[ay\pp +by\p+cy=0 \qquad \text{we need 2 linearly independent solutions}\]
\centerline{Try \uline{ansatz} $y=e^{rt}$ \hspace{3em} $\rightarrow$\hspace{3em}$e^{rt}\times \underbrace{(ar^2+br+c)}_{\text{characterisitc polynomial}} = 0$}

\[r_{1,2} = \frac{-b\pm\sqrt{b^2-4ac} }{2a}\]
\enum{\item Two distinct real roots: $\underbrace{b^2-4ac}_{\text{discriminant}}>0$ \\$y_h = c_1e^{r_1 t} + c_2 e^{r_2 t}$\vfill
\item Repeated real roots: discriminant $= 0$ \hspace{3em} \student{$r=\frac{-b}{2a}$}  \vfill \item Complex conjugate roots: discriminant $<0$\hspace{2em} \student{$\sqrt{\text{negative \#}}$}}
}


\subsection{Case 2: repeated root}

\slide[Repeated real root ($r_1=r_2=r$) \hfill $y_h=c_1y_1 + c_2y_2$]{
Straighforward solution \[y_1=e^{rt}\]  with \[r = \frac{-b\pm\sqrt{b^2-4ac} }{2a} = \frac{-b}{2a}\]
We need another solution that is linearly independent of $y_1$
\student{
\vfill

\vfill
Lets try \[y_2=q(t) y_1(t)\]
Unique choice \[q(t)=Ct \qquad \Rightarrow \qquad y_2(t) = te^{rt}\]
}
}
\slide[Proof that $y_2=te^{rt}$]{\vspace{-2.5em}
\[ay\pp + by\p+cy=0 \qquad \text{with } b^2-4ac=0\quad \Rightarrow \quad r_{1,2}=r=\frac{-b}{2a} \]
\centerline{$$}\vspace{-1.75em}
\algn{\text{Try: } y_2 &= q(t)e^{rt}, \quad 
y_2\p = q'e^{rt}+rte^{rt} \\
y_2\pp &= q''e^{rt} + 2rq'e^{rt}+ r^2te^{rt} 
\intertext{plug these into the ODE}
&a\left(  q''e^{rt} + 2rq'e^{rt}+ r^2te^{rt}  \right) + b \left( q'e^{rt}+rte^{rt}\right) + c q e^{rt} = 0 \\
& a q''e^{rt} +\left( 2ar +b \right)q'e^{rt} +\underbrace{ (ar^2+br+c)}_{\text{char. poly.}=0}qe^{rt} =0 
\intertext{sub in $r=\frac{-b}{2a}$} 
&aq''e^{rt} +\underbrace{\left( \cancel{2a}\frac{-b}{\cancel{2a}} +b \right)}_{0}e^{rt} =0 \\
& aq''e^{rt} = 0 \quad \Rightarrow  \quad  q''=0 \quad \Rightarrow  \quad  q(t) = Ct+D }
$D=0$ due to linear independence between $y_2$ and $y_1$
}

\slide[Constant Coefficient Homogeneous 2$^{nd}$ Order ODE]{\vspace{-1em}
\[ay\pp +by\p+cy=0\]
\centerline{Try \uline{ansatz} $y=e^{rt}$ \hspace{3em} $\rightarrow$\hspace{3em}$e^{rt}\cdot \underbrace{(ar^2+br+c)}_{\text{characterisitc polynomial}} = 0$}

\[r_{1,2} = \frac{-b\pm\sqrt{b^2-4ac} }{2a}\]
\enum{\item Two distinct real roots: $\underbrace{b^2-4ac}_{\text{discriminant}}>0$ \\$y_h = c_1e^{r_1 t} + c_2 e^{r_2 t}$\vfill
\item Repeated real roots: discriminant $= 0$ \\$y_h = c_1e^{r t} + c_2 t e^{rt}$ \vfill \item Complex conjugate roots: discriminant $<0$}
}


\slide[Solve the IVP: $y\pp+4y\p+4y=0 \qquad \rarray{y(0)=2\\y\p(0)=0}$ ]{

\student{\algn{r_{1,2} &= \frac{-4\pm\sqrt{16-4\cdot 4}}{2} = \frac{-4\pm 0}{2} =,-2\\
y_h&=c_1e^{-2t}+c_2te^{-2t} \intertext{initial conditions:}
y(0)=2&=c_1\\
y\p(0)=0&=-2c_1 + c_2(e^{-2t}-2te^{-2t})\evalat{t=0}{} \\&= -4 +c_2 \qquad \Rightarrow \qquad c_2=4\\
\Aboxed{y(t) &= 2 e^{-2t} + 4 t e^{-2t}}
}
}

}


\subsection{Case 3: complex conjugate roots}
\slide[Review: Complex Numbers]{
\twomini[.9]{.5}{.5}{
Square root of a negative number:
\vfill
Suppose $w>0$
\student{
\algn{\ucover{\sqrt{-w}} & \ucover{=} i\sqrt{w}\\
i&=\text{imaginary unit}\\\\
i \times i &= -1}
}
\vspace{4em}
}{
Suppose $a,b\in \mathbb{R}$\vfill
Complex Number: $z=a+ib$\\
Complex Conjugate: $\bar{z} =a-ib $

\student{\algn{\ucover{\frac{z+\bar{z}}{2}} & \ucover{=}  \frac{2a}{2}\; =  a = \text{Re}(z)\qquad \\ \\
\ucover{\frac{z-\bar{z}}{2i}} & \ucover{=} \frac{2ib}{2i}=  b = \text{Im}(z) \qquad
}}

\vspace{3em}
}

}
\slide[Complex roots ($b^2-4ac<0$ )  \hfill $y_h=c_1y_1 + c_2y_2$]{
Roots are given by:\algn{r_1&=\lambda + i \mu & \text{where } i =\sqrt{-1} \\
r_2&=\lambda - i \mu}
\student{\vspace{-.75em}\[\lambda = \frac{-b}{2a},\qquad \mu =\frac{\sqrt{4ac-b^2}}{2a}\]}\vfill
The two functions $y_1=e^{(\lambda +i\mu )t}$ \& $y_2=e^{(\lambda  - i\mu) t}=\bar{y}_1$ are solutions.\vfill
\student{What is the exponential of a complex number?\vfill Euler's formula:\[e^{\pm i \alpha} = \cos \alpha \pm i \sin \alpha\]}

}
\settoggle{covered}{false}
\slide{\vspace{-1.5em}
\student{
\algn{y_{1,2}=e^{(\lambda  \pm i\mu ) t} &= e^{\lambda t} e^{\pm i \mu t} \\ &=  \underbrace{e^{\lambda t}}_{\text{Real}} \underbrace{[\underbrace{ \cos(\mu t)}_{\text{Real}} \pm\underbrace{ i \sin (\mu t)}_{\text{Imaginary}}]}_{\text{Complex Conjugates}}&y_1=\bar{y}_2&}

We don't want imaginary solutions
\vfill
\algn{\tilde{y}_1 & = \frac{y_1+y_2}{2}=\text{Re}(y_1) =e^{\lambda t} \cos(\mu t)   \\
\tilde{y}_2&=\dfrac{y_1-y_2}{2i}=\text{Im}(y_1)=e^{\lambda t} \sin (\mu t)
}
\vfill
}
}

\slide[Complex roots ($r_{1,2}=\lambda\pm i \mu$)]{
The functions $y_1=e^{\lambda t} \cos (\mu t)$ and $y_2=e^{\lambda t} \sin (\mu t)$ are linearly independent real solutions. 

\student{
\vfill
Sketch the two functions if you are not convinced.\vfill
General solution: $y_h = c_1e^{\lambda t} \cos (\mu t) + c_2 e^{\lambda t} \sin (\mu t)$}

}

\slide[Find the general solution to: $y\pp+6y=0 $ ]{
\student{\algn{r_{1,2} &= \frac{\pm\sqrt{-4\cdot 6}}{2} = \pm \sqrt{-6} = \pm i\sqrt{6} \\
y_h&=c_1 \cos \left(\sqrt{6}t\right) + c_2 \sin\left( \sqrt{6}t\right) 
}
}
}

\slide[Solve the IVP: $y\pp+2y\p+5y=0 \qquad \rarray{y(0)=1\\y\p(0)=-1}$ ]{
\student{\algn{r_{1,2} &= \frac{-2\pm\sqrt{4-4\cdot 5}}{2} = \frac{-2\pm \sqrt{-16}}{2}=-1\pm \frac{\sqrt{16}}{2} i = -1\pm 2i \\
y_h&=e^{-t} \left(c_1 \cos \left(2t\right) + c_2 \sin\left( 2t\right) \right)\intertext{initial conditions:}
y(0)=1&=c_1\\
y\p(0)=-1&= -c_1+ \left(-2 c_1 \sin \left(0\right) + 2 c_2 \cos\left( 0 \right) \right) = -c_1+2c_2\\
-1&=-1+2c_2  \qquad \Rightarrow  \qquad  c_2=0\\\\
\Aboxed{y(t) &= e^{-t}\cos (2t)}
}
}
}

\slide[Qualitative Behaviour: complex roots]{\vspace{-1em}
Three subcases:

\student{
\enum{
\item  $\lambda < 0\quad\Rightarrow\quad$ Exponentially decaying oscillations.
\tikzplot{.1}{10}{.75}{.75}{t}{}{
\draw[domain=0:10, samples=300,  thick, color=blue] plot ({\x}, {0.75*exp(-.2*\x)*sin(360*\x)});
\draw[domain=0:10, samples=300,  thick, color=red] plot ({\x}, {0.75*exp(-.2*\x)*cos(360*\x)});
\draw[domain=0:10, samples=300,  thick, dashed] plot ({\x}, {0.75*exp(-.2*\x)})  node[left, xshift=-4cm, yshift=.4cm]{$e^{\lambda t}$};
\draw[domain=0:10, samples=300,  thick, dashed] plot ({\x}, {-0.75*exp(-.2*\x)}) node[left, xshift=-4cm, yshift=-.4cm]{$-e^{\lambda t}$};
}
\vfill
\item   $\lambda = 0 \quad\Rightarrow\quad$ Sustained periodic oscillations.
\tikzplot{.1}{10}{.75}{.75}{t}{}{
\draw[domain=0:10, samples=300,  thick, color=blue] plot ({\x}, {.72*sin(360*\x)});
\draw[domain=0:10, samples=300,  thick, color=red] plot ({\x}, {.72*cos(360*\x)});
}
\vfill
\item  $\lambda > 0\quad\Rightarrow\quad$ Exponentially growing oscillations.
\tikzplot{.1}{10}{.75}{.75}{t}{}{
\draw[domain=0:10, samples=300,  thick, color=blue] plot ({\x}, {0.1*exp(.2*\x)*sin(360*\x)});
\draw[domain=0:10, samples=300,  thick, color=red] plot ({\x}, {0.1*exp(.2*\x)*cos(360*\x)});
\draw[domain=0:10, samples=300,  thick, dashed] plot ({\x}, {0.1*exp(.2*\x)})  node[left, xshift=-4cm, yshift=-.15cm]{$e^{\lambda t}$};
\draw[domain=0:10, samples=300,  thick, dashed] plot ({\x}, {-0.1*exp(.2*\x)}) node[left, xshift=-4cm, yshift=.15cm]{$-e^{\lambda t}$};
}
}
}
}

\slide[Summary]{
\itmz{
\item For linear ODEs:
\subitem{Pick an ansatz (e.g., $e^{rt}$) \item Write down the characteristic equation
\item Find the roots \subitem{If you don't have enough functions, make a new one by multiplying by $t$}}
\vfill
\item Write down the general solution according to the roots
\subitem{ Real and distinct $\Rightarrow y_h = c_1e^{r_1t} + c_2 e^{r_2t}$ }
\subitem{ Real and repeated $\Rightarrow y_h = c_1e^{rt} + c_2 te^{rt} $ }
\subitem{ Complex $\Rightarrow y_h= c_1e^{\lambda t} \cos (\mu t) + c_2 e^{\lambda t} \sin (\mu t)$ }
\vfill
\item Fit the constants $c_1$ and $c_2$ to the initial conditions
}
}


\end{document}