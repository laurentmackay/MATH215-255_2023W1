\documentclass[11pt, dvipsnames, handout]{beamer}
\newtoggle{full}
\settoggle{full}{true}

\newtoggle{covered}
\settoggle{covered}{false}

\newtoggle{presentable}
\settoggle{presentable}{false}

\newtoggle{dualscreen}
\settoggle{dualscreen}{false}

\usepackage{pgfplots}
%\pgfplotsset{compat = newest}

\usepackage{pgfpages}

\setbeamertemplate{note page}{\pagecolor{yellow!5}\vfill \insertnote \vfill}
\usepackage{collect}
\definecollection{notes}
\newcounter{notestaken}

\usepackage{xpatch}

\usepackage{ulem}

\usepackage[framemethod=tikz]{mdframed}

\usepackage{scalerel}
\usepackage{calc}

%\usepackage{enumitem}
\setlength\fboxsep{.2em}

\usepackage{graphicx} % Allows including images
\usepackage{booktabs} % Allows the use of \toprule, \midrule and \bottomrule in tables

\xpatchcmd{\itemize}
  {\def\makelabel}
  {\setlength{\itemsep}{0.65 em}\def\makelabel}
  {}
  {}


\xpatchcmd{\beamer@enum@}
  {\def\makelabel}
  {\setlength{\itemsep}{0.65 em}\def\makelabel}
  {}
  {}


%\makeatletter
%\renewcommand{\itemize}[1][]{%
%  \beamer@ifempty{#1}{}{\def\beamer@defaultospec{#1}}%
%  \ifnum \@itemdepth >2\relax\@toodeep\else
%    \advance\@itemdepth\@ne
%    \beamer@computepref\@itemdepth% sets \beameritemnestingprefix
%    \usebeamerfont{itemize/enumerate \beameritemnestingprefix body}%
%    \usebeamercolor[fg]{itemize/enumerate \beameritemnestingprefix body}%
%    \usebeamertemplate{itemize/enumerate \beameritemnestingprefix body begin}%
%    \list
%      {\usebeamertemplate{itemize \beameritemnestingprefix item}}
%      {%
%        \setlength\topsep{1em}%NEW
%        \setlength\partopsep{1em}%NEW
%        \setlength\itemsep{1em}%NEW
%        \def\makelabel##1{%
%          {%
%            \hss\llap{{%
%                \usebeamerfont*{itemize \beameritemnestingprefix item}%
%                \usebeamercolor[fg]{itemize \beameritemnestingprefix item}##1}}%
%          }%
%        }%
%      }
%  \fi%
%  \beamer@cramped%
%  \raggedright%
%  \beamer@firstlineitemizeunskip%
%}
%
%
%
%
%
%\makeatother

%\setlist[beamer@enum@]{topsep=1 em}
%\let\origcheckmark\checkmark %screw you dingbat
%\let\checkmark\undefined %screw you dingbat
%\usepackage{dingbat} 
%\let\checkmark\origcheckmark %screw you dingbat






%\usepackage{fontawesome}

\usepackage{mathtools}
\usepackage{etoolbox, calculator}

\usepackage{xcolor}
\usepackage{tikz}
\usetikzlibrary{arrows.meta}
\usetikzlibrary{calc}
\usepackage[nomessages]{fp}
\usepackage{transparent}
\usepackage{accsupp}
%\usepackage{color, xcolor}

%colorblind-friendly palette
%\definecolor{dblue}{RGB}{51,34,136}
\definecolor{lblue}{RGB}{136,204,238}
%\definecolor{green}{RGB}{17,119,51}
\definecolor{tan}{RGB}{221,204,119}
%\definecolor{mauve}{RGB}{204,102,119}

\usepackage{tcolorbox}



\usepackage{xifthen}
\usepackage{nicefrac}
\usepackage{amsmath}
\usepackage{amsthm}
\usepackage{amssymb}
\theoremstyle{definition}
\newtheorem*{define}{Definition}
\newtheorem*{recall}{Recall}


\DeclareMathOperator{\tr}{tr}

\usepackage{multicol}
%\setlength{\columnsep}{1cm}

\usepackage{tablists, amsmath,vwcol, cancel, polynom}
\usetikzlibrary{shapes, patterns, decorations.shapes}
%\usepackage{tikzpeople}
\tikzstyle{vertex}=[shape=circle, minimum size=2mm, inner sep=0, fill]
\tikzstyle{opendot}=[shape=circle, minimum size=2mm, inner sep=0, fill=white, draw]

% common math quick commands
\newcommand{\nicedd}[2]{\nicefrac{\text{d}#1}{\text{d}#2}}
\newcommand{\dd}[2]{\dfrac{\text{d}#1}{\text{d}#2}}
\newcommand{\pd}[2]{\dfrac{\partial #1}{\partial#2}}
\renewcommand{\d}[1]{\text{d}#1}
\newcommand{\ddn}[3]{\dfrac{\text{d}^{#3}#1}{\text{d}#2^{#3}}}
\newcommand{\pdn}[3]{\dfrac{\partial^{#3}#1}{\partial#2^{#3}}}
\newcommand{\p}[0]{^{\prime}}
\newcommand{\pp}[0]{^{\prime\prime}}
\newcommand{\op}[2][\text{L}]{#1 \left[ #2 \right]}

\newcommand{\lap}[1]{\mathcal{L}\left\{#1\right\}}
\newcommand{\lapinv}[1]{\mathcal{L}^{-1}\left\{#1\right\}}
\newcommand{\lapint}[1]{\int_0^\infty e^{-st}#1dt}
\newcommand{\evalat}[2]{\Big|_{#1}^{#2}}

\newcommand{\paren}[1]{ \left( #1 \right)}

\newcommand{\haxis}[4][\normcolor]{\draw[#1, <->] (-#2,0)--(#3,0) node[right]{$#4$}; }


\newcommand{\axis}[4]{\draw[\normcolor, <->] (-#1,0)--(#2,0) 
node[right]{$x$};
\draw[help lines, <->] (0,-#3)--(0,#4) node[above]{$y$};}

\newcommand{\laxis}[6]{\draw[<->] (-#1,0)--(#2,0) 
node[right]{$#5$};
\draw[ <->] (0,-#3)--(0,#4) node[above]{$#6$};}
\newcommand{\xcoord}[2]{
	\draw (#1,.2)--(#1,-.2) node[below]{$#2$};}
\newcommand{\textnode}[3]{
	\draw (#1,#2) node[below]{$#3$};}
	
\newcommand{\nxcoord}[2]{
	\draw (#1,-.2)--(#1,.2) node[above]{$#2$};}
\newcommand{\ycoord}[2]{
	\draw (.2,#1)--(-.2,#1) node[left]{$#2$};}
\newcommand{\nycoord}[2]{
	\draw (-.2,#1)--(.2,#1) node[right]{$#2$};}
\newcommand{\dlim}{\displaystyle\lim}
\newcommand{\dlimx}[1]{\displaystyle\lim_{x \rightarrow #1}}
\newcommand{\stickfig}[2]{
	\draw (#1,#2) arc(-90:270:2mm);
	\draw (#1,#2)--(#1,#2-.5) (#1-.25,#2-.75)--(#1,#2-.5)--(#1+.25,#2-.75) (#1-.2,#2-.2)--(#1+.2,#2-.2);}	

%\newcounter{example}
%\setcounter{example}{1}
%\newcounter{preFrameExample}
%\AtBeginEnvironment{frame}{\setcounter{preFrameExample}{\value{example}}}
%\newcommand{\ex}[1]{
%	 \setcounter{example}{\value{preFrameExample}}
%	 \textcolor{green}{\small\fbox{Example \arabic{example}: #1}}\\[8pt]
%	\stepcounter{example}}
%\newcommand{\exans}[1]{
%	\SUBTRACT{\value{preFrameExample}}{1}{\n}
%	 \textcolor{green}{\small\fbox{Solution \n: #1}}\\[8pt]}
\mode<presentation> {

% The Beamer class comes with a number of default slide themes
% which change the colors and layouts of slides. Below this is a list
% of all the themes, uncomment each in turn to see what they look like.


\usetheme{CambridgeUS}
\usecolortheme[named=black]{structure}


\newcommand{\studentcolor}[0]{ForestGreen}
\newcommand{\normcolor}[0]{NavyBlue}
\newcommand{\alertcolor}{Red}

\setbeamercolor{normal text}{fg=\normcolor}
\setbeamercolor{frametitle}{fg=\normcolor}
\setbeamercolor{section in head/foot}{fg=Black, bg=Gray!20}
\setbeamercolor{subsection in head/foot}{fg=Green!70!Black, bg=Gray!10}
\setbeamercolor{alerted text}{fg=\alertcolor}
\setbeamerfont{alerted text}{series=\bf}
\setbeamertemplate{enumerate items}[default]
\setbeamercolor{enumerate item}{fg=\normcolor}

\setbeamertemplate{footline} % To remove the footer line in all slides uncomment this line
%\setbeamertemplate{footline}[page number] % To replace the footer line in all slides with a simple slide count uncomment this line

\setbeamertemplate{navigation symbols}{} % To remove the navigation symbols from the bottom of all slides uncomment this line
}

\newcommand{\alertbox}[1]{\tcbox[on line, colframe=\alertcolor, colback=White, left=2pt,right=2pt,top=2pt,bottom=2pt]{\usebeamercolor*{normal text}#1}}


\newcommand{\startstu}{\setbeamercolor{normal text}{fg=\studentcolor}\usebeamercolor*{normal text}\setbeamercolor{enumerate item}{fg=\studentcolor}\usebeamercolor*{enumerate item}}
\newcommand{\stopstu}{\setbeamercolor{normal text}{fg=\normcolor}\usebeamercolor*{normal text}\setbeamercolor{enumerate item}{fg=\normcolor}\usebeamercolor*{enumerate item}}

\newcommand{\takenote}[1]{ \begin{collect}{notes}{}{}{}{}  #1  \end{collect}  \addtocounter{notestaken}{1}} %\ifthenelse{\value{notestaken}>0}{\hrulefill\\}{}

\makeatletter
\newcommand{\cover}{\alt{\beamer@makecovered}{\beamer@fakeinvisible}}
\newcommand{\ucover}[1]{\iftoggle{full}{}{\beamer@endcovered}\stopstu#1\startstu\iftoggle{full}{}{\beamer@startcovered}}
\makeatother

\newcommand{\skippause}{ \addtocounter{beamerpauses}{-1}}
\newcommand{\blockpres}{ \skippause \pause }

\newcommand{\studentify}[1]{\startstu #1  \stopstu }
\newcommand{\student}[1]{\iftoggle{full}{ \pause  \studentify{#1} }{\iftoggle{covered}{\studentify{#1}}{\cover{  #1 }}}}
\newcommand{\cstudent}[1]{\student{\begin{center} #1 \end{center}}}
\newcommand{\fullonly}[1]{\iftoggle{full}{ #1}{}}
\newcommand{\presentonly}[1]{\iftoggle{presentable}{ #1}{}}

\usepackage{xparse}
\usepackage{xifthen}

% shortcuts for commonly-used presentation elements
%\NewDocumentCommand{\slide}{o m}
% {\IfValueTF{#1}{\begin{frame}[t]{#1}}{\begin{frame}[t]} #2 \end{frame}}

\newtoggle{iscovered}

\newcommand{\slide}[2][]{%
%\setcounter{notestaken}{0}
\takenote{#2} 
%\ifthenelse{\equal{#1}{}}{\begin{frame}[t]}{\begin{frame}[t]{#1}} #2 \ifthenelse{\value{notestaken}>0}{ \note{\includecollection{notes}}}{} \end{frame}%
\ifthenelse{\equal{#1}{}}{\begin{frame}[t]}{\begin{frame}[t]{#1}} #2 \iftoggle{covered}{\settoggle{iscovered}{true}}{\settoggle{iscovered}{false}}  \note{ \iftoggle{iscovered}{}{\settoggle{covered}{true}} #2 \iftoggle{iscovered}{}{\settoggle{covered}{false}} } \end{frame}%
%\setcounter{notestaken}{0}
}
\newcommand{\defn}[2][]{%
 \setcounter{listcounter}{0}%
\ifthenelse{\equal{#1}{}}{\begin{block}{Definition}}{\begin{block}{#1 :}}%
 #2 \vspace{0.25em} \ifthenelse{\value{listcounter}>0}{\skippause}{} \pause \end{block}%
}



\newcommand{\arr}[2]{\begin{array}{#1}#2\end{array}}
\newcommand{\mat}[2]{\left[\arr{#1}{#2}\right]}
\newcommand{\carray}[1]{\arr{c}{#1}}
\newcommand{\larray}[1]{\arr{l}{#1}}
\newcommand{\rarray}[1]{\arr{r}{#1}}
\newcommand{\colvec}[1]{\mat{c}{#1}}

\newcommand{\itmz}[1]{\addtocounter{listcounter}{1} \begin{itemize}#1 \end{itemize} }
\newcommand{\subitem}[1]{\addtocounter{listcounter}{1} \begin{itemize} \item #1 \end{itemize}}
%
\newcommand{\enum}[1]{\addtocounter{listcounter}{1} \begin{enumerate} #1  \end{enumerate}  }


\newcommand{\algnlbl}[1]{\begin{align}#1  \end{align}} 
\newcommand{\algn}[1]{\begin{align*}#1  \end{align*}} 
\newcommand{\lgn}[1]{ \action<+->{#1} }
\newcommand{\slgn}[1]{\iftoggle{full}{\action<+->{ \startstu #1 \stopstu}}{ \cover{ #1 } } \takenote{$#1$}}

\newcommand{\chckmrk}{\alert{\checkmark}}

\usepackage{pifont}
\newcommand{\xmark}{\alert{\text{\large \ding{55}}}}

\newcommand{\return}[0]{\raisebox{.5ex}{\rotatebox[origin=c]{180}{$\Lsh$}}}
\usepackage{pbox}
%\newcommand{\ex}[1]{\rotatebox[origin=c]{10}{\uline{ex}}:$\;$\pbox[t][][b]{0.9\linewidth}{#1}}
\newcommand{\ex}[1]{\uline{ex}:$\;$\pbox[t][][t]{0.9\linewidth}{#1}}
\newcommand{\eg}[1]{e.g.,$\;$\pbox[t][][t]{0.9\linewidth}{#1}}
\newcommand{\tikzplot}[8][]{%
\begin{tikzpicture}

\begin{scope}[]%
\clip(-#2,-#4) rectangle (#3,#5);%
#8%
\end{scope}%
\laxis{#2}{#3}{#4}{#5}{#6}{#7}%
#1
\end{tikzpicture}%
}


\newcommand{\cancelslide}[1]{%
\begingroup%
\setbeamertemplate{background canvas}{%
\begin{tikzpicture}[remember picture,overlay]%
\draw[line width=2pt,red!60!black] %
  (current page.north west) -- (current page.south east);%
\draw[line width=2pt,red!60!black] %
  (current page.south west) -- (current page.north east);%
\end{tikzpicture}}%
#1%
\endgroup%
}
\renewcommand{\CancelColor}{\color{red}}
\newcommand{\twocols}[3][0.5]{\begin{columns}\begin{column}{#1\textwidth}#2\end{column}\hspace{1em}\vrule{}\hspace{1em}\begin{column}{#1\textwidth}#3\end{column}\end{columns}}

\newcommand{\twomini}[5][1]{\calculatespace \begin{minipage}[t]{\columnwidth}\begin{minipage}[][#1\contentheight][t]{#2\columnwidth}#4\end{minipage}\hfill\begin{minipage}[][#1\contentheight][t]{#3\columnwidth}#5\end{minipage}\end{minipage}}

\newcommand{\threemini}[7][1]{\calculatespace \begin{minipage}[t]{\columnwidth}\begin{minipage}[][#1\contentheight][t]{#2\columnwidth}#5\end{minipage}\hfill\begin{minipage}[][#1\contentheight][t]{#4\columnwidth}#6\end{minipage}\hfill\begin{minipage}[][#1\contentheight][t]{#3\columnwidth}#7\end{minipage}\end{minipage}}


\newcounter{listcounter}
\setcounter{listcounter}{0}



\newif\ifsidebartheme
\sidebarthemetrue

\newdimen\contentheight
\newdimen\contentwidth
\newdimen\contentleft
\newdimen\contentbottom
\makeatletter
\newcommand*{\calculatespace}{%
\contentheight=\paperheight%
\ifx\beamer@frametitle\@empty%
    \setbox\@tempboxa=\box\voidb@x%
  \else%
    \setbox\@tempboxa=\vbox{%
      \vbox{}%
      {\parskip0pt\usebeamertemplate***{frametitle}}%
    }%
    \ifsidebartheme%
      \advance\contentheight by-1em%
    \fi%
  \fi%
\advance\contentheight by-\ht\@tempboxa%
\advance\contentheight by-\dp\@tempboxa%
\advance\contentheight by-\beamer@frametopskip%
\ifbeamer@plainframe%
\contentbottom=0pt%
\else%
\advance\contentheight by-\headheight%
\advance\contentheight by\headdp%
\advance\contentheight by-\footheight%
\advance\contentheight by4pt%
\contentbottom=\footheight%
\advance\contentbottom by-4pt%
\fi%
\contentwidth=\paperwidth%
\ifbeamer@plainframe%
\contentleft=0pt%
\else%
\advance\contentwidth by-\beamer@rightsidebar%
\advance\contentwidth by-\beamer@leftsidebar\relax%
\contentleft=\beamer@leftsidebar%
\fi%
}
\makeatother


\iftoggle{dualscreen}{\setbeameroption{show notes on second screen=right}}{}


\begin{document}
\section{Lecture 11}
\subsection{Introduction}

\settoggle{covered}{true}

\slide[Recall: Superposition  for Linear Homogeneous ODEs]{
Suppose the linearly independent functions $y_1(t)$ and $y_2(t)$ both independently solve a 2$^{\text{nd}}$ order \uline{linear} homogeneous ODE\[ \op{y}=0\] then \[y_h =c_1y_1(t) + c_2y_2(t)\] is the general solution to the ODE.

\vfill

Proof:
\algn{\op{c_1y_1+c_2y_2} &\overset{\text{Linearity 1}}{=}  \op{c_1y_1}+\op{c_2y_2}\\
&\overset{\text{Linearity 2}}{=} c_1\op{y_1} + c_2 \op{y_2}\\
&\quad\;\; =c_1\cdot 0 + c_2 \cdot 0  = 0
}
}

\slide[Superposition for Linear Inhomogeneous ODEs]{\vspace{-1.05em}
Suppose the linearly independent functions $y_1(t)$ and $y_2(t)$ both solve a 2$^{\text{nd}}$ order \uline{linear} homogeneous ODE\algn{ &\op{y}=0 &&\text{(1)}  \intertext{ and a particular solution $y_p(t)$ solves the linear inhomogeneous ODE } &\op{y_p} = f(t) \neq 0 && \text{(2)} } then $y_g=\underbrace{c_1y_1 +c_2y_2}_{\arr{c}{\text{\small homogeneous or}\\ \text{\small complementary}\\\text{\small solution}}}+ \quad y_p$ is the general solution to Eq. (2).\vspace{-.5em}

Proof:\student{
\algn{\op{c_1y_1+c_2y_2+y_p} &\overset{\text{Linearity 1}}{=}  \op{c_1y_1}+\op{c_2y_2}+\op{y_p}\\
&\overset{\text{Linearity 2}}{=}\underbrace{ c_1\op{y_1} + c_2 \op{y_2}}_{0}+\underbrace{\op{y_p}}_{f(t)} = f(t)
}}
For proof of uniqueness of $y_p$, see DiffQs \S 2.5.1
}

\slide[Solution Structure for Linear Inhomogeneous Problems]{\vspace{-1em}
The general solution to \[\op{y(t)} = f(t) \qquad\text{with }f(t)\neq0 \]
where $\op{\cdot}$ is linear is \uline{always} of the form\[y_g(t) = y_h(t) + y_p(t)\] where $y_h$ solves $\op{y}=0$ the associated homogeneous problem.\vfill 
\ex{Second order $\op{y} \neq 0$ with $y(0)=y_0$, $y'(0)=v_0$ }
\student{\algn{
y_h &= c_1y_1(t)+c_2y_2(t)\\
y(0) &= y_0 = c_1y_1(0) + c_2y_2(0) + y_p(0) \\
y'(0) &= v_0 = c_1y_1'(0) + c_2y_2'(0) + y_p'(0) 
}\vfill
\centerline{Solve for $c_1$ and $c_2$. Don't forget to include $y_p$.}
}
}



\subsection{Method of Undetermined Coefficients}
\slide[How to find $y_p$?\hfill Method of Undetermined Coefficients]{\vspace{-1.05em}
Suppose we get: \[ay''+by'+cy=0 \quad \Rightarrow \quad  \text{guess }y_h=e^{rt}\]
Suppose we get: \[ay''+by'+cy=f(t) \]
\student{
Guess $y_p = $ similar to $f(t)$
\vfill
This approach works for: \itmz{
\item $f(t)=$ polynomial func.
\item $f(t)=$ exponential func.
\item $f(t)=$ $\sin$ or $\cos$
\item additive combinations of above
\item multiplicative combinations of above
}
}
}


\slide[How to find $y_p$?\hfill Method of Undetermined Coefficients]{\vspace{-1.05em}

Suppose we get: \[ay''+by'+cy=Dt+E \quad \student{\Rightarrow \quad  \text{guess }y_p=Ft+H}\]
\student{Plug in guess to find $F$ and $H$:
\algn{&y_p'=F, \qquad y_p''=0\\
&bF+cFt+cH = Dt+E \qquad \text{(one eq. two unknowns)}\intertext{group by powers of $t$ to obtain two eqs.}
&\uline{t^1}: \quad cFt =Dt \qquad \Rightarrow  F= D/c\\
&\uline{t^0}: \quad bF + cH =D \qquad \\
 &\qquad\;\; b\frac{D}{c} + cH = E \qquad \Rightarrow H= \frac{E}{c}-b\frac{D}{c^2}
}\vfill
Works as long as $c\neq0$...otherwise the algebra is impossible.}

}

\slide{\ex{ Find the particular solution to $y''+2y'+2y=2t^2-2$.}
\vfill
\student{
Guess: $y_p=At^2+Bt+C$
\algn{y_p'=2At+B,\quad y_p''=2A& \\
2A+2(2At+B)+2(At^2+Bt+C)&=2t^2-2
}
group by powers of $t$:
\algn{\uline{t^2:}& \quad 2At^2  = 2t^2 \quad \Rightarrow A=1\\
\uline{t^1:} &\quad 4\cancelto{1}{A}t + 2Bt = 0\\
&\quad  4+2B = 0 \quad \Rightarrow B=-2\\
\uline{t^0:} &\quad 4\cancelto{1}{A}t + 2Bt = -2\\
&\quad  2\cancelto{1}{A}+2\cancelto{-2}{B} +2C = -2\\
&\quad 2-4+2C = -2 \quad \Rightarrow C=0
&\quad \Aboxed{y_p=t^2-2t}
}

}
}

\slide{\ex{ Solve $y''+2y'+2y=2t^2-2$ with $y(0)=1, y'(0)=-3$ . }
\student{
\algn{y &= c_1 y_1 + c_2 y_2 + \underbrace{y_p}_{t^2-2t}\\
y_{1,2} &= e^{rt} & r&=\frac{-2\pm\sqrt{4-8}}{2}\\
&&&=-1\pm i\\
y(t) &=  e^{-t}\paren{c_1\cos(t) +c_2\sin(t)} +t^2-2t\intertext{initial  conditions:}
y(0)&=1=c_1 \quad \Rightarrow c_1 = 1\\
y'(t)&=-e^{t} \paren{c_1\cos(t) +c_2\sin(t)} \\&+ e^{t}  \paren{c_2\cos(t) - c_1\sin(t)}  + 2t - 2\\\\
y'(0)& = -3 = -c_1 +c_2 -2 \Rightarrow c_2 = 0\\
\Aboxed{y(t)& = e^{-t}\cos(t)+t^2-2t}
}}


}



\slide[Mathematical Resonance \& Undetermined Coefficients]{ Given an inhomogeneous linear DE $\op{y}= f(t) \neq 0$, we say that
\uline{mathematical resonance} occurs when $f(t)$ (or its derivatives) has the same form as $y_h$.\\~\\
\centerline{ result =  impossible algebra}
\vfill
\algn{\uline{ex:} \quad x\pp + x &= \sin \omega t \\ y_h &= c_1\cos( t) + c_2\sin( t)}
\twomini[.27]{.5}{.5}{\algn{&\uline{\omega  \neq 1}\quad  \student{\text{ (no problem)}}\\
y_p &= A\cos(\omega t)+B\sin(\omega t) }}{\algn{&\uline{\omega  = 1} \quad  \student{\text{ (resonance)}}\\
y_p &= At\cos( t)+Bt\sin( t) }}
\vfill
\alert{Trick:} Multiply the na{\"i}ve guess for $y_p(t)$ by $t^k$ where $k$ is large enough to ensure that $y_p$ is not of the same form as $y_h$.
}

\slide[Method of Undetermined Coefficients \hfill - \hfill Resonance]{
$ay\pp+by\p+cy=f(t)$

\enum{\item Solve the associated homogeneous eqn. to get $y_h(t)$:\student{\centerline{$y_h=c_1y_1 + c_2y_2$}}\vspace{-1.25em}
\item Determine the "family of functional forms"  for $f(t)$ by differentiating:\vfill
\ex{} $\cos(3t)\quad \student{\left\{ \cos(3t), \sin(3t)\right\} \Rightarrow y_p = A \cos(3t) + B\sin(3t)}$ \vfill
\ex{} $t^2e^{-t}\quad \student{\left\{ t^2e^{-t}, te^{-t}, e^{-t}\right\} \Rightarrow y_p = A t^2e^{-t} + B te^{-t} + C e^{-t}}$\vfill
\ex{} $t\sin2t \quad \student{\left\{ t\sin2t, t\cos2t, \sin2t, \cos2t \right\} \Rightarrow y_p = \cdots$}
\vfill
If the family for $f(t)$ has $N$ members, $y_p(t)$ \uline{must} have $N$ L.I. terms.
\vfill
\item Check for that none of the family members look like $y_1(t)$ or $y_2(t)$.
\vfill
\student{Multiply family members by $t$ until there is no more resonance.}

}

}%end slide

\slide[Practice spotting resonance]{\twocols{\enum{[(1)]
\item $y\p+6y=\cos t +t^2$  \vspace{-1em} \student{\algn{y_h &=c_1e^{-6t} \\ \text{family} &=\left\{\cos t,\sin t, t^2, t, 1\right\} \\ y_p =& A\cos t +B \sin t \\& +C t^2+Dt+E}} \vspace{-2.5em}
\item $y\pp=t^2$ \student{\hfill $y_h=c_1+c_2t$\algn{\text{family} &=\left\{t^2,\uline{t},\uline{1}\right\} \\ y_p &= At^2+Bt^3+Ct^4}}\vspace{-2em}
\item $y\pp + 3y\p+2y=5e^{-t}$ \vspace{-1em} \student{\algn{y_h &=c_1e^{-t}+c_2e^{-2t}\\ \text{family} &=\left\{\uline{e^{-t}}\right\} \\ y_p &= Ate^{-t}}} \vspace{-2.5em}
}}{\enum{ 
\item[(4)] $y\pp+2y\p+y=12e^{-t}$\vspace{-1em} \student{\algn{y_h &=c_1e^{-t}+c_2 t e^{-t} \\ \text{family} &=\left\{\uline{e^{-t}}\right\} \\ y_p &= At^2e^{-t}}} \vspace{.5em}
\item[(5)] $y\pp+6y\p=\cos t +t^2$ \vspace{-1em} \student{\algn{y_h &=c_1e^{-6t} + c_2 \\ \text{family} &=\left\{\cos t,\sin t, t^2, t, \uline{1}\right\} \\ y_p =& A\cos t +B \sin t \\& +C t^2+Dt+Et^3}} \vspace{-2.5em}
}}
}%end slide

\settoggle{covered}{false}
\slide{Find the general solution of $y\pp+5y\p+4y=e^{-4t}$
\student{\algn{r_{1,2}&=\frac{-5\pm\sqrt{25-16}}{2} = \frac{-5\pm3}{2} = -1, -4\\
y_h &= c_1 e^{-t} +\underbrace{c_2e^{-4t}}_{\propto f(t)}\\
\text{Try: }\quad y_p&=Ate^{-4t}\\
y_p\p &= A \left( e^{-4t} - 4te^{-4t}\right)\\
y_p\pp &= -Ae^{-4t} -4A \left(e^{-4t}  - 4te^{-4t}\right)\\
&= -8Ae^{-4t} + 16 Ate^{-4t}
}
}
}
\slide{
\student{\algn{\intertext{plug into DE:}
-8Ae^{-4t} +16 Ate^{-4t} &+ 5Ae^{-4t} -20Ate^{-4t} + 4Ate^{-4t} =e^{-4t}\\
(-8+5)Ae^{-4t} +& (20-20)te^{-4t} =e^{-4t}\\
-3Ae^{-4t} &=e^{-4t}\\\\
A&=-\frac13\\\\
y &=c_1e^{-4t}+c_2e^{-t} -\frac13te^{-4t}
}}}


\slide{Find the general solution of $y\pp+4y\p+4y=e^{-2t}$
\student{\algn{r_{1,2}&=\frac{-4\pm\sqrt{16-16}}{2} = -2\\
y_h &=\underbrace{c_1e^{-2t}}_{\propto f(t)} + c_2te^{-2t}\\
\text{Try: }\quad y_p&=At^2e^{-2t}\\
y_p\p &= A \left( 2te^{-2t} - 2t^2e^{-2t}\right)\\
y_p\pp &= 2A \left( e^{-2t} - 2te^{-2t}\right) -2A  \left( 2te^{-2t} - 2t^2e^{-2t}\right)\\
&=4 At^2 e^{-2 t} -8 At e^{-2 t} +2 A e^{-2 t}
}
}}
\slide{\student{\algn{ \intertext{plug into DE:}
4 At^2 e^{-2 t} -8 At e^{-2 t}& +2 A e^{-2 t} +  8A  te^{-2t} -  8At^2e^{-2t} + 4At^2e^{-2t} =e^{-2t}\\
(-8+8)At^2e^{-2t} +& (-8+8)te^{-2t}+2Ae^{-2t} =e^{-2t}\\
2Ae^{-2t} &=e^{-2t}\\\\
2A&=1 \qquad \Rightarrow A=\frac12\\\\
y &= c_1 e^{-2t} +c_2te^{-2t} + \frac12 t^2e^{2t}
}
}
}


\subsection{Method of Undetermined Coefficients}
\slide[\begin{minipage}{.5\textwidth}Method of Undetermined\\ Coefficients:\end{minipage}\hfill\begin{minipage}{.34\textwidth}$\rarray{ay\pp+by\p+cy=f(t)\\y_g(t)=y_p(t)+y_h(t)}$\end{minipage}]{

\small
\vfill
\begin{table}\setlength\extrarowheight{5pt}
\begin{tabular}{|c|c|}
\hline 
Form of function $f(t)$ & Guess for $y_{p}\left(t\right)$\tabularnewline
\hline 
\hline 
$\sum_{j=0}^{N}d_{j}t^{j}$ & $\sum_{j=0}^{N}A_{j}t^{j}$\tabularnewline
\hline 
$e^{\lambda t}$ & $Ae^{\lambda t}$\tabularnewline
\hline 
$\sin( \omega t)$ or $\cos  (\omega t)$ & $A\sin( \omega t)+B\cos( \omega t)$\tabularnewline
\hline 
$e^{\lambda t}\sin \omega t$ or $e^{\lambda t}\cos \omega t$ & $e^{\lambda t}A\sin \omega t+e^{ \lambda t}B\cos \omega t$\tabularnewline
\hline 
Additive combinations of above & Additive combinations of above\tabularnewline
\hline 
Multiplicative combinations of above & Multiplicative combinations of above\tabularnewline
\hline 
Part of the homogeneous solution {\tiny Note}\footnotemark& $Atf(t) \text{ or } At^2f(t)$\tabularnewline
\hline 
Anything else & You are out of luck\tabularnewline
\hline 
\end{tabular}
\end{table}\vfill

\footnotetext[1]{Note: This corresponds to resonance. }
\footnotetext[2]{Note: $a$, $b$, $c$, $d_j$ , $A_j$ , $A$, and $B$ are all constants in the above table}


}

\end{document}