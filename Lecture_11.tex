\input{notes.tex}

\iftoggle{dualscreen}{\setbeameroption{show notes on second screen=right}}{}


\begin{document}
\section{Lecture 11}
\subsection{Introduction}

\settoggle{covered}{true}

\slide[Recall: Superposition  for Linear Homogeneous ODEs]{
Suppose the linearly independent functions $y_1(t)$ and $y_2(t)$ both independently solve a 2$^{\text{nd}}$ order \uline{linear} homogeneous ODE\[ \op{y}=0\] then \[y_h =c_1y_1(t) + c_2y_2(t)\] is the general solution to the ODE.

\vfill

Proof:
\algn{\op{c_1y_1+c_2y_2} &\overset{\text{Linearity 1}}{=}  \op{c_1y_1}+\op{c_2y_2}\\
&\overset{\text{Linearity 2}}{=} c_1\op{y_1} + c_2 \op{y_2}\\
&\quad\;\; =c_1\cdot 0 + c_2 \cdot 0  = 0
}
}

\slide[Superposition for Linear Inhomogeneous ODEs]{\vspace{-1.05em}
Suppose the linearly independent functions $y_1(t)$ and $y_2(t)$ both solve a 2$^{\text{nd}}$ order \uline{linear} homogeneous ODE\algn{ &\op{y}=0 &&\text{(1)}  \intertext{ and a particular solution $y_p(t)$ solves the linear inhomogeneous ODE } &\op{y_p} = f(t) \neq 0 && \text{(2)} } then $y_g=\underbrace{c_1y_1 +c_2y_2}_{\arr{c}{\text{\small homogeneous or}\\ \text{\small complementary}\\\text{\small solution}}}+ \quad y_p$ is the general solution to Eq. (2).\vspace{-.5em}

Proof:\student{
\algn{\op{c_1y_1+c_2y_2+y_p} &\overset{\text{Linearity 1}}{=}  \op{c_1y_1}+\op{c_2y_2}+\op{y_p}\\
&\overset{\text{Linearity 2}}{=}\underbrace{ c_1\op{y_1} + c_2 \op{y_2}}_{0}+\underbrace{\op{y_p}}_{f(t)} = f(t)
}}
For proof of uniqueness of $y_p$, see DiffQs \S 2.5.1
}

\slide[Solution Structure for Linear Inhomogeneous Problems]{\vspace{-1em}
The general solution to \[\op{y(t)} = f(t) \qquad\text{with }f(t)\neq0 \]
where $\op{\cdot}$ is linear is \uline{always} of the form\[y_g(t) = y_h(t) + y_p(t)\] where $y_h$ solves $\op{y}=0$ the associated homogeneous problem.\vfill 
\ex{Second order $\op{y} \neq 0$ with $y(0)=y_0$, $y'(0)=v_0$ }
\student{\algn{
y_h &= c_1y_1(t)+c_2y_2(t)\\
y(0) &= y_0 = c_1y_1(0) + c_2y_2(0) + y_p(0) \\
y'(0) &= v_0 = c_1y_1'(0) + c_2y_2'(0) + y_p'(0) 
}\vfill
\centerline{Solve for $c_1$ and $c_2$. Don't forget to include $y_p$.}
}
}



\subsection{Method of Undetermined Coefficients}
\slide[How to find $y_p$?\hfill Method of Undetermined Coefficients]{\vspace{-1.05em}
Suppose we get: \[ay''+by'+cy=0 \quad \Rightarrow \quad  \text{guess }y_h=e^{rt}\]
Suppose we get: \[ay''+by'+cy=f(t) \]
\student{
Guess $y_p = $ similar to $f(t)$
\vfill
This approach works for: \itmz{
\item $f(t)=$ polynomial func.
\item $f(t)=$ exponential func.
\item $f(t)=$ $\sin$ or $\cos$
\item additive combinations of above
\item multiplicative combinations of above
}
}
}


\slide[How to find $y_p$?\hfill Method of Undetermined Coefficients]{\vspace{-1.05em}

Suppose we get: \[ay''+by'+cy=Dt+E \quad \student{\Rightarrow \quad  \text{guess }y_p=Ft+H}\]
\student{Plug in guess to find $F$ and $H$:
\algn{&y_p'=F, \qquad y_p''=0\\
&bF+cFt+cH = Dt+E \qquad \text{(one eq. two unknowns)}\intertext{group by powers of $t$ to obtain two eqs.}
&\uline{t^1}: \quad cFt =Dt \qquad \Rightarrow  F= D/c\\
&\uline{t^0}: \quad bF + cH =D \qquad \\
 &\qquad\;\; b\frac{D}{c} + cH = E \qquad \Rightarrow H= \frac{E}{c}-b\frac{D}{c^2}
}\vfill
Works as long as $c\neq0$...otherwise the algebra is impossible.}

}

\slide{\ex{ Find the particular solution to $y''+2y'+2y=2t^2-2$.}
\vfill
\student{
Guess: $y_p=At^2+Bt+C$
\algn{y_p'=2At+B,\quad y_p''=2A& \\
2A+2(2At+B)+2(At^2+Bt+C)&=2t^2-2
}
group by powers of $t$:
\algn{\uline{t^2:}& \quad 2At^2  = 2t^2 \quad \Rightarrow A=1\\
\uline{t^1:} &\quad 4\cancelto{1}{A}t + 2Bt = 0\\
&\quad  4+2B = 0 \quad \Rightarrow B=-2\\
\uline{t^0:} &\quad 4\cancelto{1}{A}t + 2Bt = -2\\
&\quad  2\cancelto{1}{A}+2\cancelto{-2}{B} +2C = -2\\
&\quad 2-4+2C = -2 \quad \Rightarrow C=0
&\quad \Aboxed{y_p=t^2-2t}
}

}
}

\slide{\ex{ Solve $y''+2y'+2y=2t^2-2$ with $y(0)=1, y'(0)=-3$ . }
\student{
\algn{y &= c_1 y_1 + c_2 y_2 + \underbrace{y_p}_{t^2-2t}\\
y_{1,2} &= e^{rt} & r&=\frac{-2\pm\sqrt{4-8}}{2}\\
&&&=-1\pm i\\
y(t) &=  e^{-t}\paren{c_1\cos(t) +c_2\sin(t)} +t^2-2t\intertext{initial  conditions:}
y(0)&=1=c_1 \quad \Rightarrow c_1 = 1\\
y'(t)&=-e^{t} \paren{c_1\cos(t) +c_2\sin(t)} \\&+ e^{t}  \paren{c_2\cos(t) - c_1\sin(t)}  + 2t - 2\\\\
y'(0)& = -3 = -c_1 +c_2 -2 \Rightarrow c_2 = 0\\
\Aboxed{y(t)& = e^{-t}\cos(t)+t^2-2t}
}}


}



\slide[Mathematical Resonance \& Undetermined Coefficients]{ Given an inhomogeneous linear DE $\op{y}= f(t) \neq 0$, we say that
\uline{mathematical resonance} occurs when $f(t)$ (or its derivatives) has the same form as $y_h$.\\~\\
\centerline{ result =  impossible algebra}
\vfill
\algn{\uline{ex:} \quad x\pp + x &= \sin \omega t \\ y_h &= c_1\cos( t) + c_2\sin( t)}
\twomini[.27]{.5}{.5}{\algn{&\uline{\omega  \neq 1}\quad  \student{\text{ (no problem)}}\\
y_p &= A\cos(\omega t)+B\sin(\omega t) }}{\algn{&\uline{\omega  = 1} \quad  \student{\text{ (resonance)}}\\
y_p &= At\cos( t)+Bt\sin( t) }}
\vfill
\alert{Trick:} Multiply the na{\"i}ve guess for $y_p(t)$ by $t^k$ where $k$ is large enough to ensure that $y_p$ is not of the same form as $y_h$.
}

\slide[Method of Undetermined Coefficients \hfill - \hfill Resonance]{
$ay\pp+by\p+cy=f(t)$

\enum{\item Solve the associated homogeneous eqn. to get $y_h(t)$:\student{\centerline{$y_h=c_1y_1 + c_2y_2$}}\vspace{-1.25em}
\item Determine the "family of functional forms"  for $f(t)$ by differentiating:\vfill
\ex{} $\cos(3t)\quad \student{\left\{ \cos(3t), \sin(3t)\right\} \Rightarrow y_p = A \cos(3t) + B\sin(3t)}$ \vfill
\ex{} $t^2e^{-t}\quad \student{\left\{ t^2e^{-t}, te^{-t}, e^{-t}\right\} \Rightarrow y_p = A t^2e^{-t} + B te^{-t} + C e^{-t}}$\vfill
\ex{} $t\sin2t \quad \student{\left\{ t\sin2t, t\cos2t, \sin2t, \cos2t \right\} \Rightarrow y_p = \cdots$}
\vfill
If the family for $f(t)$ has $N$ members, $y_p(t)$ \uline{must} have $N$ L.I. terms.
\vfill
\item Check for that none of the family members look like $y_1(t)$ or $y_2(t)$.
\vfill
\student{Multiply family members by $t$ until there is no more resonance.}

}

}%end slide

\slide[Practice spotting resonance]{\twocols{\enum{[(1)]
\item $y\p+6y=\cos t +t^2$  \vspace{-1em} \student{\algn{y_h &=c_1e^{-6t} \\ \text{family} &=\left\{\cos t,\sin t, t^2, t, 1\right\} \\ y_p =& A\cos t +B \sin t \\& +C t^2+Dt+E}} \vspace{-2.5em}
\item $y\pp=t^2$ \student{\hfill $y_h=c_1+c_2t$\algn{\text{family} &=\left\{t^2,\uline{t},\uline{1}\right\} \\ y_p &= At^2+Bt^3+Ct^4}}\vspace{-2em}
\item $y\pp + 3y\p+2y=5e^{-t}$ \vspace{-1em} \student{\algn{y_h &=c_1e^{-t}+c_2e^{-2t}\\ \text{family} &=\left\{\uline{e^{-t}}\right\} \\ y_p &= Ate^{-t}}} \vspace{-2.5em}
}}{\enum{ 
\item[(4)] $y\pp+2y\p+y=12e^{-t}$\vspace{-1em} \student{\algn{y_h &=c_1e^{-t}+c_2 t e^{-t} \\ \text{family} &=\left\{\uline{e^{-t}}\right\} \\ y_p &= At^2e^{-t}}} \vspace{.5em}
\item[(5)] $y\pp+6y\p=\cos t +t^2$ \vspace{-1em} \student{\algn{y_h &=c_1e^{-6t} + c_2 \\ \text{family} &=\left\{\cos t,\sin t, t^2, t, \uline{1}\right\} \\ y_p =& A\cos t +B \sin t \\& +C t^2+Dt+Et^3}} \vspace{-2.5em}
}}
}%end slide

\settoggle{covered}{false}
\slide{Find the general solution of $y\pp+5y\p+4y=e^{-4t}$
\student{\algn{r_{1,2}&=\frac{-5\pm\sqrt{25-16}}{2} = \frac{-5\pm3}{2} = -1, -4\\
y_h &= c_1 e^{-t} +\underbrace{c_2e^{-4t}}_{\propto f(t)}\\
\text{Try: }\quad y_p&=Ate^{-4t}\\
y_p\p &= A \left( e^{-4t} - 4te^{-4t}\right)\\
y_p\pp &= -Ae^{-4t} -4A \left(e^{-4t}  - 4te^{-4t}\right)\\
&= -8Ae^{-4t} + 16 Ate^{-4t}
}
}
}
\slide{
\student{\algn{\intertext{plug into DE:}
-8Ae^{-4t} +16 Ate^{-4t} &+ 5Ae^{-4t} -20Ate^{-4t} + 4Ate^{-4t} =e^{-4t}\\
(-8+5)Ae^{-4t} +& (20-20)te^{-4t} =e^{-4t}\\
-3Ae^{-4t} &=e^{-4t}\\\\
A&=-\frac13\\\\
y &=c_1e^{-4t}+c_2e^{-t} -\frac13te^{-4t}
}}}


\slide{Find the general solution of $y\pp+4y\p+4y=e^{-2t}$
\student{\algn{r_{1,2}&=\frac{-4\pm\sqrt{16-16}}{2} = -2\\
y_h &=\underbrace{c_1e^{-2t}}_{\propto f(t)} + c_2te^{-2t}\\
\text{Try: }\quad y_p&=At^2e^{-2t}\\
y_p\p &= A \left( 2te^{-2t} - 2t^2e^{-2t}\right)\\
y_p\pp &= 2A \left( e^{-2t} - 2te^{-2t}\right) -2A  \left( 2te^{-2t} - 2t^2e^{-2t}\right)\\
&=4 At^2 e^{-2 t} -8 At e^{-2 t} +2 A e^{-2 t}
}
}}
\slide{\student{\algn{ \intertext{plug into DE:}
4 At^2 e^{-2 t} -8 At e^{-2 t}& +2 A e^{-2 t} +  8A  te^{-2t} -  8At^2e^{-2t} + 4At^2e^{-2t} =e^{-2t}\\
(-8+8)At^2e^{-2t} +& (-8+8)te^{-2t}+2Ae^{-2t} =e^{-2t}\\
2Ae^{-2t} &=e^{-2t}\\\\
2A&=1 \qquad \Rightarrow A=\frac12\\\\
y &= c_1 e^{-2t} +c_2te^{-2t} + \frac12 t^2e^{2t}
}
}
}


\subsection{Method of Undetermined Coefficients}
\slide[\begin{minipage}{.5\textwidth}Method of Undetermined\\ Coefficients:\end{minipage}\hfill\begin{minipage}{.34\textwidth}$\rarray{ay\pp+by\p+cy=f(t)\\y_g(t)=y_p(t)+y_h(t)}$\end{minipage}]{

\small
\vfill
\begin{table}\setlength\extrarowheight{5pt}
\begin{tabular}{|c|c|}
\hline 
Form of function $f(t)$ & Guess for $y_{p}\left(t\right)$\tabularnewline
\hline 
\hline 
$\sum_{j=0}^{N}d_{j}t^{j}$ & $\sum_{j=0}^{N}A_{j}t^{j}$\tabularnewline
\hline 
$e^{\lambda t}$ & $Ae^{\lambda t}$\tabularnewline
\hline 
$\sin( \omega t)$ or $\cos  (\omega t)$ & $A\sin( \omega t)+B\cos( \omega t)$\tabularnewline
\hline 
$e^{\lambda t}\sin \omega t$ or $e^{\lambda t}\cos \omega t$ & $e^{\lambda t}A\sin \omega t+e^{ \lambda t}B\cos \omega t$\tabularnewline
\hline 
Additive combinations of above & Additive combinations of above\tabularnewline
\hline 
Multiplicative combinations of above & Multiplicative combinations of above\tabularnewline
\hline 
Part of the homogeneous solution {\tiny Note}\footnotemark& $Atf(t) \text{ or } At^2f(t)$\tabularnewline
\hline 
Anything else & You are out of luck\tabularnewline
\hline 
\end{tabular}
\end{table}\vfill

\footnotetext[1]{Note: This corresponds to resonance. }
\footnotetext[2]{Note: $a$, $b$, $c$, $d_j$ , $A_j$ , $A$, and $B$ are all constants in the above table}


}

\end{document}