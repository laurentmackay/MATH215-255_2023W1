\input{notes.tex}


\iftoggle{dualscreen}{\setbeameroption{show notes on second screen=right}}{}
\usetikzlibrary{arrows}
\usetikzlibrary{decorations.markings}

\begin{document}
\section{Lecture 25}
\subsection{Preamble}


\slide[Recall: Linearization + Local Behaviour]{
The non-linear autonomous system \[\dd{}{t} \vec{x}  = \vec{f}(\vec{x}(t))\]
can often be well-approximated near a critical point $\vec{x}^*$ using
\[\dd{}{t} \vec{x}  \approx\mathbf{J}^*\vec{x} - \mathbf{J}^*\vec{x}^* \quad \text{where} \quad  \mathbf{J}^*=\mathbf{J}(\vec{x}^*) \quad \text{with}\quad  \mathbf{J}=\dd{\vec{f}}{\vec{x}}. \]
\student{

\vfill
The eigenvalues of the Jacobian matrix, evaluated at a critical point, determine the \uline{local} behaviour of solutions in some neighbourhood around the critical point.

}

}

\slide[Classification of 2D Nonlinear Critical Points]{
\begin{tabular}{|>{\centering}m{3.5cm}|>{\centering}m{3.5cm}|>{\centering}m{3.5cm}|}
\hline 
Eigenvalues & Stability & Type\tabularnewline
\hline 
\hline 
\vspace{.5em}\student{$\lambda_{1}\leq\lambda_{2}<0$}\vspace{.5em}&\student{asymptotically stable} & \student{sink (node)}\tabularnewline
\hline 
\vspace{.5em}\student{$0<\lambda_{1}\leq\lambda_{2}$}\vspace{.5em} &\student{ unstable} &\student{ source (node)}\tabularnewline
\hline 
\vspace{.5em}\student{$\lambda_{1}<0<\lambda_{2}$}\vspace{.5em} &\student{ unstable} & \student{saddle}\tabularnewline
\hline 
\vspace{.5em}\student{one of $\lambda_{1},\lambda_{2}=0$}\vspace{.5em}& \student{unclear} & \student{unclear}\tabularnewline
\hline 
\vspace{.5em}\student{$r\pm i\omega$ with $r<0$}\vspace{.5em} & \student{asymptotically stable} & \student{spiral sink}\tabularnewline
\hline 
\vspace{.5em}\student{$r\pm i\omega$ with $r>0$}\vspace{.5em} & \student{unstable} & \student{spiral source}\tabularnewline
\hline 
\vspace{.5em}\student{$r\pm i\omega$ with $r=0$}\vspace{.5em} & \student{unclear} & \student{unclear}\tabularnewline
\hline 
\end{tabular}
\vfill


\student{For the \uline{unclear} cases, we need  higher order approximations...(not covered)}

}
\subsection{Nullclines and Critical Points}
\slide[How to find critical points?]{\vspace{-1em}
Given
\[x'=f(x,y),\quad y'=g(x,y)\] we want to find the set of points $(x^*,y^*)$ where \[ f(x^*,y^*)=0 \quad \text{and} \quad g(x^*,y^*)=0.\]
\student{
Consider each constraint separately $\quad\Rightarrow\quad$ nullclines
\algn{0&=f(x,y)&0&=g(x,y)\intertext{each expression gives some number of curves (i.e. nullclines).
}}

\vspace{-2em}
Nullcline = set of points that makes a derivative zero.
\vfill
 Critical points exist at the intersection of the x- and y-nullclines.\vfill
}


}
\slide{
\ex{$x'=y$ and $y'=-x+x^2\student{=x(x-1)}$}
\student{
\algn{\text{x-nullclines: } &&\text{y-nullclines: }\\
 0&=y &0&=x(x-1)\\
y&=0&x&=0\quad \text{and} \quad x=1 \\
}
}

\vspace{-1.5em}

\centerline{\tikzplot[\xcoord{-1}{-1}\xcoord{-2}{-2}\xcoord{1}{1}\xcoord{2}{2}\ycoord{-1}{-1} \ycoord{1}{1}]{4}{4}{2}{2}{x}{y}{
\student{
\draw[black, ultra thick] (-4,0) --(4,0) node[xshift=-3em, above]{x-nullcline};
\draw[black, ultra thick, dashed] (1,-2) --(1,2) node[xshift=3em, below]{y-nullcline 2};
\draw[black, ultra thick, dashed] (0,-2) --(0,2) node[xshift=-3em, below]{y-nullcline 1};
\filldraw[black] (0,0) circle (3pt);
\filldraw[black] (1,0) circle (3pt);
}
}}

\student{\[\text{\uline{critical points:} } (0,0) \quad \text{and} \quad (1,0)\]}
}

\slide{
\ex{Find all critical points for $x'=x-x^2-2xy$, $y'=2y-2y^2-3xy$}
\student{
\algn{\text{x-nullclines: } &&\text{y-nullclines: }\\
0&=x(1-x-2y) &0&=y(2-2y-3x)\\
x=0& \;\; \text{and} \;\;\quad y=\frac12-\frac12x&y&=0\quad \text{and} \quad y=1-\frac32 x \\
}
}
\vspace{-2.5em}

\centerline{\tikzplot[
\xcoord{-1}{-\frac14}\xcoord{-2}{-\frac12}
\xcoord{1}{\frac14}\xcoord{2}{\frac12}\xcoord{3}{\frac34}\xcoord{4}{1}\xcoord{5}{\frac54}
\ycoord{-1}{-\frac12} \ycoord{1}{\frac12}\ycoord{2}{1}
]{3}{6}{1.5}{2.5}{x}{y}{
\student{
\draw[black, ultra thick, dashed] (-4,0) --(4,0);
\draw[black, ultra thick, dashed, domain=-2:5] plot ({\x},{2-3*\x/4});

\draw[black, ultra thick] (0,-2.5) --(0,4) ;
\draw[black, ultra thick] (-12,4) --(20,-4) ;
\filldraw[black] (0,0) circle (3pt);
\filldraw[black] (0,2) circle (3pt);
\filldraw[black] (4,0) circle (3pt);
\filldraw[black] (2,0.5) circle (3pt);
\draw (4,2) node[align=center]{$\frac12-\frac12x=1-\frac32x$\\$x=\frac12 \quad \Rightarrow \quad y=\frac14$}; 
}
}}

\student{\[\text{\uline{critical points:} }\quad  (0,0) , \quad (1,0),\quad (0,1), \quad \text{and} \quad \left(\nicefrac12.\nicefrac14\right)\]}

}

\slide{
\ex{Find all critical points for $x'=yx-x-x^3$, $y'=y(x^2+1-y)$}
\student{
\algn{\text{x-nullclines: } &&\text{y-nullclines: }\\
0&=x(y-1-x^2) & y&=0 \quad \text{and} \quad y=1+x^2 \\
x=0& \;\; \text{and} \;\; y=1+x^2\\
}
}
\vspace{-3em}


\centerline{\tikzplot[
\xcoord{-1}{-1}\xcoord{-2}{-2}\xcoord{1}{1}\xcoord{2}{2}\ycoord{2}{2} \ycoord{1}{1}]
{4}{4}{.5}{2.5}{x}{y}{
\student{
\draw[black, ultra thick, dashed] (-4,0) --(4,0);

\draw[black, ultra thick] (0,-2) --(0,4) ;
\filldraw[black] (0,0) circle (3pt);
\draw[black, domain=-2:2] plot ({\x},{1+\x*\x});
\draw[black, ultra thick, dashed, domain=-2:2] plot ({\x},{1+\x*\x});
}
}}

\student{\[\text{\uline{isolated critical point:} }\quad  (0,0)  \qquad \uline{\text{non-isolated critical points:}}  \quad y=1+x^2\]}
A critical point is \emph{isolated} if it is the only critical point in some small “neighborhood” of the point. 
}

\slide[Almost Linear Systems \hfill - \hfill Conditions for Linearization]{
A system is called \alert{almost linear} at a critical point $\vec{x}^*$ if
\enum{\item  the critical point is isolated and \item  the Jacobian matrix at the point is invertible.}\vfill
\student{ In such a case, the higher order terms can safely be ignored and the system behaves like its linearization close to the critical point.}\vfill

\ex{$x'=x^2$, $y'=y^2$}
\student{
Isolated fixed point at $(0,0)$.
\[\mathbf{J}=\mat{cc}{2x&0\\0&2y} \quad \Rightarrow \quad \mathbf{J}^* \mat{cc}{0&0\\0&0}\]
This Jacobian matrix cannot be inverted, so we cannot study the fixed point using a linear approximation.
}


}

\end{document}
