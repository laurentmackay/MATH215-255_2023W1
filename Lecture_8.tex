\input{notes.tex}

\iftoggle{dualscreen}{\setbeameroption{show notes on second screen=right}}{}

\usepackage{circuitikz}

\begin{document}
\section{Lecture 8}
\subsection{Introduction}

\slide[Recall:  Linear 1$^{st}$ Order ODEs \hfill $y\p +p(t) y = h(t)$]{
Operator form: $\op{y}=h(t)$
\vfill
\itmz{
\item Linear 1$^{st}$ order operator ${\text L}$
\item $h(t)$ is a forcing term  \alert{(inhomogeneity)}\vfill
\item Linear + $h(t)=0\quad \Rightarrow$ \alert{ Homogeneous ODE}\vfill
\item Linear + $h(t)\neq0\quad \Rightarrow$ \alert{ Inhomogeneous ODE}\vfill
}\vfill

Solved by method of integrating factors:
\algn{\mu(t) y(t) &= \intop \mu(t) h(t) dt + C\\
y(t)& = \frac{ \intop \mu(t) h(t) dt}{ \mu(t)} + \underbrace{\frac{C}{ \mu(t)}}_{\text{indep. of $h(t)$}}}



}
\subsection{Theory of Linear ODEs}
\slide[General Solution Structure of Linear ODEs]{
\ex{1$^{st}$ Order Initial Value Problems}
 \[y\p +p(t) y = h(t); \qquad y(0)=y_0 \]\vspace{.5em}
\vfill
\twomini[.5]{.5}{.5}{\uline{Associated Homogeneous Problem:}
\algn{y_h'+p(t)y_h&=0\\\ \Rightarrow \quad y_h = \frac{C}{ \mu(t)} }
}{
\uline{ Inhomogeneous DE:}\algn{ y(t) &= \frac{ \intop \mu(t) h(t) dt}{ \mu(t)} +y_h(t) \\\\ \student{y(t)} &\student{= \underbrace{y_p(t)}_{\text{particular part}} + \underbrace{y_h(t)}_{\text{homogeneous part}}}}\
}\vfill
\student{
\centering
All linear DEs have this type of solution structure.
\[ y(t) = \text{particular part + homogeneous part}\]
}
}

\slide[ Linear 2$^{nd}$ order ODEs]{
General DE: \student{\[y\pp +p(t)y\p +q(t)y=h(t)\]}Initial Conditions: \student{\[y(t_0)=y_0,\qquad y\p(t_0)=v_0\]}

Focus on  constant coefficient case first\vfill
\student{
\uline{simplest case}: homogeneous \[ ay\pp +by\p+cy=0\]
\vfill
\centering
We want intuition for how homogenoues solutions work.

}\vfill
For non-constant coefficients: method of reduction of order (DiffyQs \S 2.1)

}%end slide

\slide[Superposition Principle for Linear Homogeneous ODEs]{
Suppose the functions $y_1(t)$ and $y_2(t)$ both independently solve a \uline{linear} homogeneous ODE\[ \op{y}=0\] then \[y =c_1y_1(t) + c_2y_2(t)\] is also a solution to the same ODE.\vfill
\student{
Proof:
\algn{\op{c_1y_1+c_2y_2} &\overset{\text{Linearity 1}}{=}  \op{c_1y_1}+\op{c_2y_2}\\
&\overset{\text{Linearity 2}}{=} c_1\op{y_1} + c_2 \op{y_2}\\
&\quad\;\; =c_1\cdot 0 + c_2 \cdot 0  \\&\quad\;\; = 0
}}
}


\slide[Completeness of solutions]{
Suppose that $y_1(t)$ and $y_2(t)$ both solve a 2$^{nd}$ order linear homogeneous ODE\[\op{y}=0\]
If $y_1$ and $y_2$ are linearly independent functions, then the general solution to the homogeneous problem is\[y_h(t)=c_1y_1(t)+c_2y_2(t)\]\vfill
\alert{Take home message:} 
\student{If you can find two linearly independent solutions to a homogeneous 2$^{nd}$ order linear DE, then you have found all of them}\vfill
Proof: DiffyQs \S 2.1 (Theorem 2.1.3)
}


\slide[Linear dependence of functions]{
\itmz{
\item Two functions $f(t)$ and $g(t)$ are \textbf{linearly dependent} on the interval $t \in I=[\alpha,\beta]$ if there exist a non-zero constant $k$
such that \[f(t)= kg(t)  \quad \forall t\in I\] 
\vfill
\twomini[.4]{.5}{.5}{
Linearly Dependent\\
\tikzplot{.1}{4.3}{1}{1.25}{t}{}{
\draw[red ,domain=0:2.5, samples=120,  thick] plot ({\x}, {0.16*\x*\x}) node[right]{$f(t)=4t^2$};
\draw[blue ,domain=0:2.5, samples=120,  thick] plot ({\x}, {0.04*\x*\x}) node[right, ]{$g(t)=t^2$};
\draw[black ,domain=0:2.1, samples=120,  thick] plot ({\x}, {-0.08*\x*\x}) node[right, ]{$h(t)=-2t^2$};
}
}{
Not Linearly Dependent\\
\tikzplot{.1}{4}{.65}{1.65}{t}{}{
\draw[black ,domain=0:2, samples=120,  thick] plot ({\x}, {0.06*exp(1.5*\x)}) node[right, xshift=-.2em]{$h(t)=e^{\frac{3}{2}t}$};
\draw[red ,domain=0:2.5, samples=120,  thick] plot ({\x}, {0.06*exp(\x)}) node[right, xshift=-.6em, yshift=-.7em]{$f(t)=e^t$};
\draw[blue ,domain=0:2.5, samples=120,  thick] plot ({\x}, {0.4*sin(4*deg(\x))}) node[right, yshift=-.6em, xshift=-2em]{$g(t)=\sin(t)$};
}

}
\vfill
\item If functions are not linearly dependent on $I$, then we say they are \textbf{linearly independent} on $I$.
\vfill

}
}


\subsection{The characteristic polynomial and its roots}
\slide[Solutions to $ay\pp+by\p+cy=0$]{
Lets try an \uline{ansatz} of $y(t)=e^{rt}$ ... where $r$ is unknown \student{(ansatz method)}\vspace{-.35em}
\student{\algn{a\ddn{}{t}{2} e^{rt} +b \dd{}{t} e^{rt} + ce^{rt}&=0\\
 ar^2e^{rt} + bre^{rt} + ce^{rt}&=0 &
e^{rt}\left( ar^2+br+c\right)&=0}}
Since $e^{rt} \neq 0$, the ODE can only have a solution $y(t)=e^{rt}$ if \student{\[ ar^2+br+c=0.\]} This is called the \textbf{characteristic equation}.
\vfill
Possible values of $r$: \student{\[r=r_{1,2}=\frac{-b\pm\sqrt{b^2-4ac} }{2a}\]}
}


\slide[Roots of the characteristic equation (polynomial)]{
\[r=r_{1,2}=\frac{-b\pm\sqrt{b^2-4ac} }{2a}\]
Three main cases:\student{
\enum{\item Two distinct real roots: $\underbrace{b^2-4ac}_{\text{discriminant}}>0$ \vfill
\item Repeated real roots: discriminant $= 0$ \vfill \item Complex conjugate roots: discriminant $<0$}}
}

\slide[Case 1:  distinct real roots]{
\[y_h(t)=c_1e^{r_1 t} + c_2 e^{r_2 t}; \qquad r_{1,2}=\frac{-b\pm\sqrt{b^2-4ac} }{2a}\] 
 Two major subcases:
\student{\enum{
\item $ac>0$ :\vfill
 Both roots have the same sign.\vfill
$y_1$ and $y_2$ both grow or decay exponentially.
\vfill
\item $ac<0$:
\vfill
The two roots have opposite sign.
\vfill
One solution grows exponentially, the other decays exponentially.

}}


}




\slide[Qualitative Behaviour: distinct real roots]{\vspace{-1em}
Sum of real exponential functions, three subcases:\vfill
\student{
\enum{
\item All positive roots, $0<\textcolor{blue}{r_1}<\textcolor{red}{r_2}$.\\
\tikzplot{.1}{4}{.1}{1.25}{t}{}{
\draw[red ,domain=0:4, samples=120,  thick] plot ({\x}, {0.01*exp(2*\x)});
\draw[blue ,domain=0:4, samples=120,  thick] plot ({\x}, {0.01*exp(\x)});
}
\vfill
\item  Mixed roots, $\textcolor{blue}{r_1}<0<\textcolor{red}{r_2}$.\\
\tikzplot{.1}{4}{.1}{1.25}{t}{}{
\draw[blue ,domain=0:4, samples=120,  thick] plot ({\x}, {1*exp(-2*\x)});
\draw[red ,domain=0:4, samples=120,  thick] plot ({\x}, {0.05*exp(\x)});
}
\vfill
\item All negative roots,  $\textcolor{blue}{r_1}<\textcolor{red}{r_2}<0$.
\\
\tikzplot{.1}{4}{.1}{1.25}{t}{}{
\draw[red ,domain=0:4, samples=120,  thick] plot ({\x}, {exp(-1*\x)});
\draw[blue ,domain=0:4, samples=120,  thick] plot ({\x}, {exp(-2*\x)});
}
}
}
}
\subsection{Summary}

\slide[Summary]{
\itmz{
\item Superposition of Homogeneous solutions
\[y_h = c_1 y_1 + c_2 y_2\]\vfill
\item Linear independence of homoegeneous solutions
\[y_g=c_1y_1 +c_2y_2  \quad \Rightarrow \quad y_1 \neq k y_2; \quad k=\text{constant}\]
$y_g=$ general solution, solves ALL scenarios where a solution exists.\vfill
\item $ay\pp +by\p+cy=0$\vfill
\centerline{Try \uline{ansatz} $y=e^{rt}$ \hspace{3em} $\rightarrow$\hspace{3em}$e^{rt}\cdot \underbrace{(ar^2+br+c)}_{\text{characterisitc polynomial}} = 0$}
\vfill
\[r_{1,2} = \frac{-b\pm\sqrt{b^2-4ac} }{2a}\]
}

}




\begin{comment}


\slide{\huge \vfill \centering Where do 2$^{nd}$ order linear ODEs\\come from? \vfill}
\subsection{Some example systems}
\slide[Spring-dashpot system:]{
\twomini{.4}{.65}{\includegraphics[width=\columnwidth]{images/spring_dashpot.pdf}}{\vfill
$x(t)=$ displacement from rest position\\
$f(t)=$ applied force
\vfill
Newton's 2$^{nd}$ Law:
\algn{ F &= ma\\
 - kx - \beta \dd{x}{t} +f(t) &= m\ddn{x}{t}{2}\\\\ mx\pp+\beta x\p+kx&=f(t)}\vspace*{\fill} }
}


\slide[Torsional motion of a weight on a twisted shaft:]{
\twomini{.32}{.65}{\includegraphics[width=\columnwidth]{images/twisted_shaft.pdf}}{
\vfill\[ I\ddn{\theta}{t}{2} +c \dd{\theta}{t} + k\theta = T(t)\]\vspace*{\fill}
}
}

\slide[L-R-C series circuits:]{
\twomini{.45}{.55}{
\begin{circuitikz} \draw
       (0,-1.5)   to[vsourcesin, l=\textcolor{black}{$E(t)$}] (0,1.5)
	[black] -- (0.5,1.5)
        to[L, l=$L$, black] (1.5,1.5) -- (2.5,1.5)
        to[C, l=$C$, black] (2.5,0) -- (2.5,-1.5)
        to[R, l=$R$, black] (.5,-1.5) -- (0,-1.5) ;
    \end{circuitikz}
}{\vfill
Q=charge on capacitor\\~\\
$\dd{Q}{t}$=current in circuit\\~\\
$E(t)$ = applied voltage\\
\vfill
Kirchoff's Laws:
\[ L\ddn{Q}{t}{2} +R \dd{Q}{t} + \frac1CQ = E(t)\]\vspace*{\fill}
}
}

\slide[Small oscillations of a pendulum:]{
\twomini{.45}{.55}{\includegraphics[width=\columnwidth]{images/pendulum.pdf}}{
\vfill\[ mL^2\ddn{\theta}{t}{2}=-cL \dd{\theta}{t} -mgL\theta + F(t) \]\vspace*{\fill}
}
}

\slide[Equivalence of Problems]{
These 4 physical systems are modelled identically by:\[Ay\pp+By\p+Cy=D(t)\]
\vfill
Constants have different physical meaning (\& units)
\small\vfill
\begin{tabular}{|>{\centering}m{2cm}|>{\centering}m{2cm}|>{\centering}m{2cm}|>{\centering}m{2cm}|>{\centering}p{2cm}|}
\hline 
\textbf{System} & \textbf{A} & \textbf{B} & \textbf{C} & \textbf{D}\tabularnewline
\hline 
\hline 
\textbf{Spring Dashpot} & Mass & Damping Coeff. & Spring Constant & Applied Force\tabularnewline
\hline 
\textbf{Pendulum} & Mass x (Length)$^{2}$ & Damping x Length & Gravitational Moment & Applied Moment\tabularnewline
\hline 
\textbf{Series Circuit} & Inductance & Resistance & Capacitance$^{-1}$ & Imposed Voltage\tabularnewline
\hline 
\textbf{Twisted Shaft} & Moment of Inertia & Damping & Elastic Shaft Constant & Applied Torque\tabularnewline
\hline 
\end{tabular}

}

\slide[General Linear 2nd Order DE’s]{\vspace{-1em}
\[
\ddn{y}{t}{2} + p(t)\dd{y}{t} +q(t)y=h(t)  
\]

\itmz{\item $h(t)$ represents the "forcing" term, an external influence.
\student{\subitem{$h(t)=0$: solutions tell you the intrinsic behaviour of the system \vfill\subitem{e.g., pull on a spring-mass system, and then let it go} \vfill \item $h(t)\neq0$: solutions tell you the response of the system to forcing \vfill \subitem{e.g., periodically hit a spring-mass system}}}\vfill
\item $p(t)$ and $q(t)$ represent the intrinsic properties of the physical system.
\student{\subitem{Often consant, but not always \subitem{ e.g., an aging spring could be modelled by $q=q(t)$.}}}
}
}
\subsection{Finding solutions - the ansatz method}
\slide{
\ex{$y\pp+3y\p=0$}
\student{We can reduce the order by integrating!
\algn{y\p+3y &= C &
\paren{e^{3t}y}\p &=ce^{3t} \\
e^{3t}y &=\underbrace{\frac{C}{3}}_{c_1}e^{3t} +c_2 &
y &=\textcolor{red}{c_1}+\textcolor{blue}{c_2e^{-3t}}\\
}
This is a special ODE where we can simplify by integrating.\vfill In general, solve using the \uline{ansatz} $y=e^{rt}:$
\algn{y\p &=re^{rt}&
y\pp&=r^2e^{rt}\\
r^2e^{rt}+3re^{rt}&=0&
r(r+3) &=0 \\&&\Rightarrow r=\textcolor{red}{0},\textcolor{blue}{-3}\\
y&=\textcolor{red}{c_1e^0 }+\textcolor{blue}{c_2e^{-3t}} &=\textcolor{red}{c_1 }+\textcolor{blue}{c_2e^{-3t}}
}
}
}%end slide

\slide{
\ex{Solve $y\pp+3y\p=0$ with $y(0)=2$ and $y\p(0)=-2$}
\student{

\algn{ y_g(t) &= c_1+c_2e^{-3t} & y(0) &= c_1 +c_2  = 2\\ y_g\p (t) &= -3c_2e^{3t} & y\p(0) &=-3c_2 = -2\\\\
c_2&=\frac23\\
c_1 +\frac23&=2=\frac63& c_1&=\frac43\\\\
y(t) &= \frac43 + \frac23e^{-3t}
}

}
}%end slide


\slide{
\ex{Find the general solution to $-2y\pp+5y\p+3y=0$}\\~\\
Hint: guess an ansatz $y=e^{rt}$

\student{
\algn{y\p &=re^{rt}&
y\pp&=r^2e^{rt}\\
-2r^2e^{rt}+5re^{rt}+3e^{rt}&=0&
-2r^2+5r+3 &=0 \\ r_{1,2}&=\frac{-5\pm\sqrt{25+24}}{-4} &&=\frac{-5\pm\sqrt{49}}{-4}\\
&=\frac{-5\pm 7}{-4} & &=\frac{2}{-4}, \frac{-12}{-4}\\
&=-\frac12, 3\\
y(t)&=c_1e^{-\frac{t}{2}}+c_2e^{3t}}
}
}%end slide

\slide[Summary]{
\itmz{\item  Linear 2$^{nd}$ order ODEs are useful for modelling many physical systems\vfill
\item  2$^{nd}$ order $\Rightarrow$ 2 Initial Conditions\vfill
\item  Linear 2$^{nd}$ order homogeneous ODEs with constant coefficients \[ ay\pp +by\p+cy=0\]
 Ansatz method:
\enum{\item Guess $y(t)=e^{rt}$ \item Plug guess into the ODE \item Find (up to) 2 values of $r$  }\[y_h=c_1e^{r_1t}+c_2e^{r_2t}\]
}
}
\end{comment}


\end{document}