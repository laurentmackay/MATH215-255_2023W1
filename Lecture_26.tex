\input{notes.tex}


\iftoggle{dualscreen}{\setbeameroption{show notes on second screen=right}}{}
\usetikzlibrary{arrows}
\usetikzlibrary{decorations.markings}
\settoggle{covered}{true}
\begin{document}
\section{Lecture 26}
\subsection{Preamble}
\slide[Introduction]{

Almost all nonlinear systems cannot be solved analytically. Instead, we can use local stability analysis to gain qualitative infomation about solutions.
\vfill
\enum{\item ~ \student{Classify the stability and type of all fixed points\subitem{Fixed points can lose stability, collide, and/or dissapear/appear as system parameters are changed.}}\vfill

\item ~\student{Draw phase plane sketches to get a geometric picture \subitem{Use the fact that derivatives switch sign when trajectories cross nullclines}}}
\vfill
}


\subsection{Lamps vs. Lasers}
\slide[Lamps]{
Lamps work at a quantum level  by pumping  molecules into an excited state, such that they release photons (light particles) as they return back down to the ground state.\vfill
\twomini[.5]{.5}{.5}{
\centerline{\includegraphics[width=.7\columnwidth]{images/Jablonski_lamp.pdf}}
}{
\vspace{3em}
\centerline{\includegraphics[width=.8\columnwidth]{images/incoherent.png}}
}\vfill
The excitation/emission process is random, producing photons with uncorrelated phases and orientation. This is what we call \uline{incoherent} light.
}

\slide[Lasers \hfill \small (\textbf{L}ight \textbf{A}mplification by \textbf{S}timulated \textbf{E}mission of \textbf{R}adiation)]{\vspace{-1em}
Lasers work through a process called \textbf{stimulated emission}, photon emission from one molecule triggers the emission of a photon by its neighbour with \uline{similar phase and orientation}.\vspace{1.5em}
\twomini[.5]{.65}{.35}{
\centerline{\includegraphics[width=\columnwidth]{images/Jablonski_laser.pdf}}
}{
\vspace{3em}
\centerline{\includegraphics[width=.9\columnwidth]{images/coherent.png}}
}\vspace{1.5em}


If enough stimulated emission happens, we get an emission cascade of photons with highly correlated phase and orientation. This results in \uline{coherent} light beams that can travel long distances with minimal spread.
}

\subsection{Lamps vs. Lasers}
\slide{
A simple model for a laser considers two variables: \enum{\item $N(t)$ = \student{\# of ``ordinary'' excited molecules }\item $n(t)$ = \student{\# of ``lasing'' molecules that will emit photons coherently.}}
\vfill
The dynamics of the two variables are given by 
\vspace{.5em}
\student{\algn{n' &= nN-n\\ N' &= -nN-N+p}}
\vspace{.5em}

where $p$ is a ``pumping'' parameter that quantifies the rate of excitation.
\vfill
see Laser Physics by  Miloni and Eberly (1988) for more details.
}

\slide{
\ex{Find all critical points for the system: }
\[ n' = nN-n,\quad N' = -nN-N+p \]
\vfill
\student{
\algn{\text{\uline{n-nullclines:}}& & \text{\uline{N-nullclines:}}&\\
0&=n(N-1) &0&=-N(1+n)+p\\
n=0&\quad \text{and}\quad  N=1 &N&=\frac{p}{1+n}\intertext{intersections}
n=0\;\; \&\;\; N&=\frac{p}{1+n}\Rightarrow N=p\\
N=1\;\; \& \;\; N&=\frac{p}{1+n}\Rightarrow  n =p-1\\
&\qquad \qquad \Rightarrow N=1
}
\vfill
\[\uline{\text{critical points:}} \quad \underbrace{(0,p)}_{\text{lamp}}, \quad \text{and} \quad \underbrace{(p-1, 1)}_{\text{laser}}\]
}
}




\slide{\vspace{-.5em}
\ex{Given the simple laser model}
\[ n' = nN-n,\quad N' = -nN-N+p, \]
find a condition on $p$ that ensures that the laser state is the unique stable steady state.\vspace{-0.5em}
\student{
\[\mathbf{J} = \mat{cc}{N-1 & n \\ -N & -n-1}\]
\vspace{-2.5em}

\algn{
\text{\uline{lamp:}}& \quad (0,p) & \text{\uline{laser:}}& \quad (p-1,1)\\
\mathbf{J}^* &= \mat{cc}{p-1 & 0 \\ -p & -1} & \mathbf{J}^* &= \mat{cc}{0 & p-1 \\ 1 & -p}
\intertext{Find the eigenvalues, i.e., $\det({J}^*-\lambda \mathbf{I}) =0$ }
(p-1-\lambda)(-1-\lambda)&=0 & -\lambda(-p-\lambda)&+1-p=0\\
\lambda&=p-1, -1 &\lambda^2-p\lambda+1-p&=0\\
&&\lambda = \frac{-p}2 \pm & \frac{\sqrt{(p-2)^2}}{2}&\\
&&=-1,1&-p
}
}
}

\slide{\vspace{-.5em}
\ex{Given the simple laser model}
\[ n' = nN-n,\quad N' = -nN-N+p, \]
find a condition on $p$ that ensures that the laser state is the unique stable steady state.\vspace{-0.5em}


\algn{
\text{\uline{lamp:}}& \quad (0,p) & \text{\uline{laser:}}& \quad (p-1,1)\\
\lambda &= p-1, -1 &\lambda &= -1,1-p
}
\vfill
\student{

Both steady states have an eigenvalue of -1, independent of $p$.
\vfill
For $p<1$, the lamp state is a stable  node and the laser state is a saddle (unstable).
\vfill
For $p>1$, the lamp state is a saddle and the laser state is stable node.

}
}

\slide[Laser Threshold]{
The laser and lamp steady states collide at $p=1$ and exchange their stability.
\centerline{
\tikzplot[\xcoord{2}{1}]{.1}{6}{1}{4}{p}{n^*}{
\draw[ultra thick, black] (0,0) -- (2,0);\draw[ultra thick, black, dashed] (2,0) -- (6,0);
\draw[Black] (1.25,.25) node{lamp};
\draw[ultra thick, black] (2,0) -- (6,2);\draw[ultra thick, black, dashed] (0,-1) -- (2,0);
\draw[Black] (4,1.5) node{laser};
\draw[black,->] (5,0.5)--(5,1.25);\draw[black,->] (5,2.5)--(5,1.75);
\draw[black,->] (.5,.75)--(.5,.25);\draw[black,->] (.5,-.65)--(.5,-.25);
}
}

}

\settoggle{covered}{false}

\subsection{Phase Plane Analysis of Limit Cycles}

\slide[Your heart beats cyclically]{\vspace{-1em}
The Sinoatrial (SA) node is the main clock.
\vfill
\twomini[.56]{0.475}{0.475}{
\centerline{\includegraphics[width=.55\columnwidth]{images/heart electrical system.png}}
\tiny source: https://www.hopkinsmedicine.org/health/conditions-and-diseases/anatomy-and-function-of-the-hearts-electrical-system

}{
\centerline{\includegraphics[width=\columnwidth]{images/pacemaker ap.jpg
}}
\tiny source: https://step1.medbullets.com/cardiovascular/108016/sa-node-action-potential

}
\vfill
\centerline{ These oscillations must be robust to perturbations, otherwise you would die :( }


\vfill
\student{Such periodic solutions are called stable limit cycles \subitem{ i.e., a cycle (or closed loop) that the nonlinear system approaches in the limit $t\to \infty$ }}
}


\slide[Limit Cycles]{
A limit cycle is an isolated closed trajectory. Isolated means that neighboring trajectories are not closed; they spiral either toward or away from the limit cycle.
\vfill
\student{\centerline{Spiral trajectories $\Rightarrow$ complex conjugate eignevalues.}}
\vfill
\centerline{\includegraphics[width=0.8\columnwidth]{images/limit_cycles.png}}
\vfill
\student{\centerline{Stable limit cycles form around unstable spiral fixed points}}
}

\slide[Van der Pol Oscillator]{

The Van der Pol Oscillator is a simple model that exhibits limit cycles. Its dynamics are given by \student{\[x'  = y - \frac13 x^3+x,\quad y'=a-x  \]} where $a$ is a forcing parameter.
\vfill
Find the nullclines for this model.
\student{
\algn{\text{\uline{x-nullcline:}}& & \text{\uline{y-nullcline:}}&\\
y&=\frac13 x^3-x & x&=a}
}

}
\slide[Van der Pol Oscillator]{
The Van der Pol Oscillator has a \uline{unique} fixed point at
\student{\[x=a, \quad  y = \frac13 a ( a^2-3) \]}
with eigenvalues given by
\student{\[\lambda = \frac12 \left(1 - a^2 \pm \sqrt{ a^4 - 2 a^2-3}\right) .\]

These are complex conjugate eigenvalues that become stable for $|a|>1$.
\vfill
Unstable spiral for $|a|<1$.
\vfill
Stable spiral for $|a|>1$.

}
}

\slide{
Use the nullclines to sketch a solution trajectory for \[x'  = y - \frac13 x^3+x,\quad y'=a-x  \] with a=-1.5 starting at (0,-0.5)


\centerline{
\tikzplot[\xcoord{2}{1}\xcoord{-2}{-1}\ycoord{2}{1}\ycoord{-2}{-1}]{5}{5}{2.5}{2.5}{x}{y}{
\draw[ultra thick, domain=-2.5:2.5, smooth] plot ({2*\x},{2*\x*\x*\x/3-2*\x}) ;
\draw[] (3,2) node{x-nullcline};
\draw[ultra thick, dashed] (-3,-3)--(-3,3);
\draw[] (-1.95,-1.5) node{y-nullcline};
\filldraw[] (-3,0.75) circle (3pt);
\draw[] (0,-1) node[opendot]{};
\student{
\draw(1.5,-2) node[align=center]{$x'<0$\\$y'<0$};
\draw(-3.6,-2) node[align=center]{$x'<0$\\$y'>0$};
\draw(-4,1.5) node[align=center]{$x'>0$\\$y'>0$};
\draw(-1,2) node[align=center]{$x'>0$\\$y'<0$};
}
}
}

}




\slide{
Use the nullclines to sketch a solution trajectory for \[x'  = y - \frac13 x^3+x,\quad y'=a-x  \] with a=0.5 starting at (0,-0.25)


\centerline{
\tikzplot[\xcoord{2}{1}\xcoord{-2}{-1}\ycoord{2}{1}\ycoord{-2}{-1}]{5}{5}{2.5}{2.5}{x}{y}{
\draw[ultra thick, , domain=-2.5:2.5, smooth] plot ({2*\x},{2*\x*\x*\x/3-2*\x}) ;
\draw[] (3.2,2) node{x-nullcline};
\draw[ultra thick, dashed, ] (1,-3)--(1,3);
\draw[] (1.95,-2.3) node{y-nullcline};
\filldraw[] (1,-0.916667) circle (3pt);
\draw[] (0,-0.5) node[opendot]{};

\node[rectangle, draw ,align=center] (r) at (-2,-1.8)  {$x'<0$\\$y'>0$};
\node[rectangle, draw ,align=center] (r) at (3.6,-1.8) {$x'<0$\\$y'<0$};
\node[rectangle, draw ,align=center] (r) at (-4,1.5) {$x'>0$\\$y'>0$};
\node[rectangle, draw ,align=center] (r) at (2.5,1){$x'>0$\\$y'<0$};

}
}\vfill
Note: the solution trajectory should approach a stable limit cycle
}



\end{document}
