\input{notes.tex}


\iftoggle{dualscreen}{\setbeameroption{show notes on second screen=right}}{}

\usepackage{empheq}


\begin{document}

\section{MATH 215/255}
\slide[Land Acknowledgement]{\vfill
UBC's Point Grey Campus is located on the traditional, ancestral, and unceded territory of the Musqueam. The land it is situated on has always been a place of learning for the Musqueam people, who for millennia have passed on culture, history, and traditions from one generation to the next. I encourage you to learn more at UBC's Indigenous Portal, \url{http://indigenous.ubc.ca}, and on the Musqueam's website, \url{http://musqueam.bc.ca}.\vfill
}

\slide[Who am I?]{
\vfill
\begin{itemize}
\item \textbf{My name:} Laurent MacKay (he/him/his)
\item \textbf{Position:} Postdoctoral Research Fellow \subitem{\textbf{My research:} Mathematical Biology (applied differential equations)}
\item \textbf{Office (student) hours:} TBD: \url{https://www.when2meet.com/?21070285-YbEHX}
\item \textbf{Contact:} Canvas messaging system or email: lmackay@math.ubc.ca
\end{itemize}
\vfill
}
\subsection{Syllabus Overview}
\slide[What will we learn?]{
\vfill
How to solve Ordinary Differential Equations!
\begin{itemize}
\item Analytically
\item Numerically with MATLAB
\end{itemize}\vfill
}

\slide[Course Structure]{
\textbf{Every week:}
\begin{itemize}
\item Attend class!
\begin{itemize}
\item I'll do some exercises, you do others with neighbours (bring something to write with)
\item Asking questions is encouraged
\item Weekly quizzes on Wednesdays!
\end{itemize}\vfill
\item I will post recordings on our Section's Canvas page, along with skeleton and annotated lecture notes \vfill
\item Check Canvas for important information including textbook, grade breakdown, assignment due dates, and other resources


\end{itemize}

}

\slide[MATH 215/255: Grade Breakdown]{
\itmz{ \item 10\% WebWork \subitem{Weekly starting Sep. 19$^{\text{th}}$.}  \vfill \item 5\% MATLAB HW Assignments \subitem{Due with Webworks 1,2,3,5,8, \& 9. \item Also starting Sep. 19$^{\text{th}}$.} \vfill \item 85\% Mastery Score \subitem{First quiz on Sep. 20$^{\tt th}$}}
}

\slide[Mastery-Based Grading]{
\begin{itemize}
    \item Course is broken down into 16 Learning Outcomes, each of which you will be given the opportunity to demonstrate `Mastery' of throughout the course during:
    \begin{itemize}
        \item Weekly in-class quiz (each testing 1-2 Learning Outcomes) \subitem{End of class on Wednesdays}
        \item In class retest 1 
        \item In class retest 2 
        \item Final exam 
    \end{itemize}\vfill
    \item Nearly perfect answer =  `Mastery'.
    \item  On the right track  =  `Progressing'
\item Very incomplete answer =`Beginning'
\end{itemize}\vfill
Your final Mastery Score is the percentage of Learning Outcomes that you have scored `Mastery' on by the end of the course.


}

\slide[Advantages/Disadvantages of Mastery-Based Grading]{
\itmz{\item Pros:
\subitem{  If you do well on all the weekly quizzes you don't have to write either In class retests or the Final exam
        \item You control your grade and can earn the grade you want
        \item Your grade will never go down}\vfill
\item Things to watch out for:
\subitem{Study for the weekly quizzes! You won't have time to obtain `Mastery' on all the Learning Outcomes during the In class retests or Final exam, so you want to obtain 'Mastery' on at least some of them during the quizzes
        \item You can't get by on part marks, you actually have to know how to solve a problem to obtain a `Mastery' score on a Learning Outcome
}
}
}


\subsection{Get MATLAB}
\slide[For next class...you will need access to MATLAB]{
\vfill
\enum{
\item Create a MathWorks account
\subitem{Go to \href{https://matlab.mathworks.com}{matlab.mathworks.com}
\item Click "No account? Create one!"
\item Enter your UBC email address and follow the instructions
\subitem{You can obtain one from  \uline{\href{https://www.myaccount.ubc.ca/myAccount/}{here}} using "Activate Student Email"}
\item Note it may take a few hours to activate your MathWorks account}\vfill
\item Use MATLAB Online
\subitem{Go to  \href{https://matlab.mathworks.com}{matlab.mathworks.com}
\item Sign in with your UBC email address and MathWorks password}
}\vfill
}

\end{document}