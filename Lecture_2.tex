\input{notes.tex}


\iftoggle{dualscreen}{\setbeameroption{show notes on second screen=right}}{}
\begin{document}
\section{Lecture 2}
\subsection{Introduction}

\slide[Recall:]{
Operator form of a DE:
\[\op{y(t)}  = f(t) \]
\vfill
For the next few weeks, we will focus on first order DEs\vfill
\[y\p = f(t,y).\]\vfill
We start with the two simple cases\vfill
\[ y\p = f(t) \quad \text{and} \quad y\p=f(y)\]
}

\subsection{Integrals as solutions}

\slide[Integrals as solutions]{\vspace{-1.25em}
\[ y\p = f(t) \quad \text{and} \quad y\p=f(y)\]
Strategy: \enum{\item Move everything that depends on $y$ to one side of the equal sign.\item Move everything that depends on $t$ to the other side. \item Integrate and isolate $y$.}\vfill
\ex{$y\p=\cos(t)$}
\student{\algn{\dd{y}{t} &= \cos(t) \\ \intop \text{d}y &= \intop \cos(t) \text{d}t \\ y(t) &=\sin(t) + C   \qquad \text{(General Solution)}}  }
}

\slide[Incorporating Initial Conditions]{
\ex{$y\p  = e^{-2t}$, with $y(0)=1$}
\student{\algn{ \dd{y}{t} &= e^{-2t} \\ 
 \intop \text{d}y &= \intop e^{-2t} \text{d}t  \\ 
y(t)  &= -\frac12 e^{-2t} + C   \qquad \text{(General Solution)} \intertext{impose the initial condition}
y(0)&= 1 = -\frac12 + C  \qquad \Rightarrow  \qquad C=1 +\frac12\\
\\ y(t)&=\frac32 - \frac12e^{-2t}   \qquad \text{(Particular Solution)}
}}
}


\slide{
\ex{$y\p = e^{2y}$, with y(1)=0}\student{
\algn{
 \dd{y}{t} &= e^{2y}
&\intop e^{-2y}\text{dy}  = \intop \text{d}t\\
-\frac12e^{-2y} & = t + C\\
e^{-2y} &= -2t-2C\\
y(t) &= -\frac12 \ln(-2t-2C) \intertext{impose the initial condition}
y(1) &= 0 = -\frac12 \ln(-2-2C) & \Rightarrow -2-2C = 1\\
&&C=-\frac32\\
y(t)&=-\frac12 \ln(-2t+3)
}
}
}
%\xcoord{2}{1}
\slide{
\ex{$y\p = e^{2y}$, with y(1)=0 \hspace{1em} $\Rightarrow$  \hspace{1em}  $y(t)=-\frac12 \ln(-2t+3)$}

\centering \vfill Solution blows-up in finite time!\vfill
\tikzplot[
\xcoord{4}{2}
\xcoord{2}{1}
\xcoord{-4}{-2}
\xcoord{-2}{-1}
\ycoord{1}{1}
\ycoord{2}{2}
\ycoord{-1}{-1}
]{5}{5}{1.5}{3}{t}{y(t)}{
\draw[domain=-2.5:1.5, thick, black, samples=500] plot ({2*\x},{-0.5*ln(-2*\x+3)});
}
\vfill
Domain of definition: \student{$t\in \left(-\infty,\frac32\right) $\vfill
Outside this domain, the solution does not exist.}
}
\subsection{Separable Equations}
\slide[Separable Equations]{\vspace{-.5em}
Suppose you are given \[\frac{dy}{dx}=f(x)g(y)\] where the functions $f$ and $g$ are known. Proceeding as before...
\vfill
\student{\algn{
\frac{\text{d}y} {g(y)} &=f(t) dt\\
\int \frac{dy}{g(y)} &=\int f(t) dt\\
\Gamma(y) &=F(t) + C\\\\
y(t) &= \Gamma^{-1} \paren{F(t) + C}
}\vfill
Works as long as $1/g(y)$ and $f(t)$ are integrable functions.
}

}

\slide[]{\ex{$y'=-ty, \quad y(0)=5$}
%Can combine +- in exponent

\student{\algn{
\intop \frac{\text{d}y}{y} &= \intop -t \text{d}t\\
\ln(y) &= -\frac{t^2}{2} +C_1\intertext{exponentiate both sides}
y &= e^{ -\frac{t^2}{2} +C_1} = e^{C_1}e^{-\frac{t^2}{2}} = C_2e^{-\frac{t^2}{2}} \intertext{impose the initial condition}
y(0) &=5 = C_2\\\\
\Aboxed{y(t) &= 5 e^{-\frac{t^2}{2}}}
}}
}

\slide[]{\ex{ $\dd{y}{x}=\frac{x^2}{y}, \quad y(0)=1$}
%Have to choose +for square root. Solution only defined on interval
\student{\algn{
\intop y\text{d}y&= \intop x^2 \text{d}x\\
\frac12 y^2(x)&=\frac13 x^3 + C\\
y&=\pm \sqrt{\frac23 x^3+C }  \intertext{impose the initial condition for both cases}
y(0)&=1=\sqrt{C } & y(0)&=1=-\sqrt{C } \\
C &= 1 & C& \text{ does not exist} \\\\
y(x)&= \sqrt{\frac23 x^3+1 } &  &\text{ for }  x>-\sqrt[3]{\nicefrac{3}{2}}
}}
}


\slide[]{\ex{$y'=y^2 \quad y(0)=1$}
%Simple example of finite blowup
\student{\algn{
\intop \frac{\text{d}y}{y^2} &= \intop dx\\
-\frac{1}{y} &= x+C\\
y(x) &= -\frac{1}{x+C}
 \intertext{impose the initial conditions}
y(0) &= 1 = -\frac1C \qquad C=-1\\
y(x) &= -\frac{1}{x-1}
}}
}


\slide[$y(x) = -\frac{1}{x-1}$]{
\tikzplot[
\xcoord{-4}{-2}
\xcoord{-2}{-1}
\ycoord{0.25}{1}
\ycoord{1}{4}
\ycoord{2}{8}
\ycoord{3}{12}
\ycoord{-1}{4}
\xcoord{4}{2}
\xcoord{2}{1}
]{5}{5}{1.5}{4}{x}{y(x)}{
\draw[domain=-2.5:.99, thick, black, samples=1000] plot ({2*\x},{-0.25/(\x-1)});
\draw[domain=1.01:3, thick, black, samples=1000] plot ({2*\x},{-0.25/(\x-1)});
\draw[dashed, black] (2,-1.5) -- (2,4);
}\vfill
Domain of definition: \student{$t\in \left(-\infty,1\right) \cup \left(1,\infty \right)$}
}


\end{document}