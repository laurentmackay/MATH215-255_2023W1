\input{notes.tex}


\iftoggle{dualscreen}{\setbeameroption{show notes on second screen=right}}{}
\usetikzlibrary{arrows}
\usetikzlibrary{decorations.markings}

\begin{document}
\section{Lecture 27}
\subsection{Preamble}
\slide[Introduction]{
There are many scenarios where catastrophic behaviour arises:\vfill
\student{
\itmz{ \item Plastic deformation of solids \item Financial collapse   \item Climate change tipping points}
}
\vfill
Two typical properties:
\vfill

\enum{
\item ~\student{Small change in a parameter produces drastic change in system equilibrium.}
\item  ~\student{Hard to reverse, cannot just reverse the small parameter change \\\centerline{ (hysteresis)} }
}
}


\subsection{Saddle-Node Catastrophes}

\slide[Simple model of a catastrophe]{
We can model a generic catastrophe by modifying the dynamics of the Van der Pol oscillator:
 \student{\[x'  = y - \frac13 x^3+x,\quad y'= \frac{x}2 - b -y  \]} where $b$ is some parameter
\vfill
Find the nullclines for this model.
\student{
\algn{\text{\uline{x-nullcline:}}& & \text{\uline{y-nullcline:}}&\\
y&=\frac13 x^3-x & y&=\frac{x}{2}-b}
}

}
\slide{The system \[x'  = y - \frac13 x^3+x,\quad y'= \frac{x}2 - b -y  \] has a critical point at (0,0) when $b=0$. Classify it.
\student{
\algn{ \mathbf{J} &= \mat{cc}{-x^2+1 & 1\\ \frac12 &-1} &\mathbf{J}^* &= \mat{cc}{1 &1 \\ \frac12 & -1}\\
\det(\mathbf{J}^*-\lambda I) &= (1-\lambda)(-1-\lambda) -\frac12=0\\
&\lambda^2-1-\frac12=0\\
&\lambda^2-\frac32=0 & \lambda&=\pm\frac{\sqrt{4\frac32}}{2}\\
&&&=\pm\sqrt{3/2} \\&&& \approx \pm 1.22 \\
&&&\text{(saddle)}}
}}


\slide{
Sketch solution trajectories for the system  \[x'  = y - \frac13 x^3+x,\quad  y'= \frac{x}2 - b -y  \] with $b=0$ and initial conditions at (-1,-1) and (1.5,-1).


\centerline{
\tikzplot[\xcoord{2}{1}\xcoord{-2}{-1}\xcoord{4}{2}\xcoord{-4}{-2}\ycoord{2}{1}\ycoord{-2}{-1}]{6}{6}{2.75}{2.75}{x}{y}{
\draw[ultra thick, domain=-2.5:2.5, smooth] plot ({2*\x},{2*\x*\x*\x/3-2*\x}) ;
\draw[ultra thick, domain=-3:3, smooth, dashed] plot ({2*\x},{\x}) ;
\draw[] (3.75,-1.15) node{x-nullcline};

\draw[] (5.1,1.5) node[align=left]{\footnotesize $\lambda_1=-0.814$\\\footnotesize $\lambda_2= -3.69$};
\draw[] (-5.1,-1.6) node[align=left]{\footnotesize $\lambda_1=-0.814$\\\footnotesize $\lambda_2= -3.69$};
\draw[] (1.1,-0.2) node[align=left]{\footnotesize $\lambda=\pm1.22$};
\draw[] (-1,-1.25) node{y-nullcline};
\filldraw[] (-2.12132*2,-1.06066*2) circle (3pt);
\filldraw[] (2.12132*2,1.06066*2) circle (3pt);
\filldraw[] (0,0) circle (3pt);
\draw[] (-2,-2) node[opendot]{};
\draw[] (3,-2) node[opendot]{};

\student{
\draw(1.5,-2) node[align=center]{$x'<0$\\$y'>0$};
\draw(-2.75,0) node[align=center]{$x'<0$\\$y'<0$};
\draw(-4,1.5) node[align=center]{$x'>0$\\$y'>0$};
\draw(2.5,0.6) node[align=center]{$x'>0$\\$y'>0$};
\draw[->, ultra thick] plot[smooth]  coordinates {(3,-2)  (2.5,-1) (2.12132*2,1.06066*2)};
\draw[->, ultra thick] plot[smooth]  coordinates {(-2,-2)  (-2.5,-1.2)  (-3.5,-1.3) (-2.12132*2,-1.06066*2)};
\draw[] (3.1,2.5) node[align=left]{stable node};
\draw[] (-3.2,-2.5) node[align=left]{stable node};
\draw[] (.5,0.75) node[align=left, rotate=30]{saddle};
}
}
}

}



\slide[Varying $b$ changes the number of nullcline intersections]{
\vspace{-1em}
\centerline{
\tikzplot[\xcoord{2}{1}\xcoord{-2}{-1}\xcoord{4}{2}\xcoord{-4}{-2}\ycoord{1}{1}\ycoord{2}{2}\ycoord{3}{3}\ycoord{-1}{-1}\ycoord{-2}{-2}]{6}{6}{3.5}{4}{x}{y}{
\draw[ultra thick, domain=-3:3, smooth] plot ({2*\x},{\x*\x*\x/3-\x}) ;

\filldraw[] (-2.12132*2,-1.06066) circle (3pt);
\filldraw[] (2.12132*2,1.06066) circle (3pt);
\filldraw[] (0,0) circle (3pt);

\filldraw[] (-2.60783*2, -3.30391) circle (3pt);
\filldraw[] (2.60783*2, 3.30391) circle (3pt);

\draw[] (2.9,-.8) node{x-nullcline};
\foreach \b in {-2,0, 2}{
\draw[ thick, domain=-3:3, smooth, dashed] plot ({2*\x},{\x/2-\b}) ;
\draw[black] (0.75,0.55-\b) node[rotate=15]{$b=\b$};


}
\student{
\foreach \b in {-1,1}{
\draw[ thick, domain=-3:3, smooth, dashed] plot ({2*\x},{\x/2+\b}) ;

}
}
}
}

}


\slide[]{
\vspace{-1em}
Sketch solution trajectories for the system  \[x'  = y - \frac13 x^3+x,\quad  y'= \frac{x}2 + 2 -y  \] with  initial conditions at (-1,-1) and (1.5,-1).
\centerline{
\tikzplot[\xcoord{2}{1}\xcoord{-2}{-1}\xcoord{4}{2}\xcoord{-4}{-2}\ycoord{1}{1}\ycoord{2}{2}\ycoord{3}{3}\ycoord{-1}{-1}]{5}{7}{2.25}{4}{x}{y}{
\draw[ultra thick, domain=-3:3, smooth] plot ({2*\x},{\x*\x*\x/3-\x}) ;

\filldraw[] (2.60783*2, 3.30391) circle (3pt);

\draw[] (5.25,1) node{x-nullcline};
\draw[] (-3.5,1.5) node{y-nullcline};
\foreach \b in { 2}{
\draw[ ultra thick, domain=-3:4, smooth, dashed] plot ({2*\x},{\x/2+\b}) ;
}
\draw[] (-2,-1) node[opendot]{};
\draw[] (3,-1) node[opendot]{};
\draw[] (6.1,2.5) node[align=left]{\footnotesize $\lambda_1=-0.898$\\\footnotesize $\lambda_2=-5.90$};
\student{
\draw(1.5,-1.5) node[align=center]{$x'<0$\\$y'>0$};
\draw(2,1) node[align=center]{$x'<0$\\$y'>0$};
\draw(-2,2.5) node[align=center]{$x'>0$\\$y'<0$};

\draw[->, ultra thick] plot[smooth]  coordinates {(3,-1)  (2.9,-.1) (2.60783*2,3.30391)};
\draw[ ultra thick] plot[smooth]  coordinates {(-2,-1)  (-2.5,0.7) (2.5,1.8) (2.60783*2,3.30391)};
\draw[] (3.5,3.5) node[align=left]{stable node};


}

}
}

}

\slide[Catastrophe and Hysteresis]{
\centerline{
\begin{tikzpicture}
\draw (0,0) node{\includegraphics[width=10cm]{images/catastrophe_bifrucation.pdf}};
\draw[black] (-1, 3) node[rotate=-5]{stable node};
\draw[black] (4.5, -1.8) node[rotate=-5]{stable node};
\draw[red] (2.3, 1.2) node[rotate=22]{saddle};
\student{
\draw[->, ultra thick] plot[smooth]  coordinates {(-0.5,2.7) (2.2,2.25)  (3.35,1.75)};
\draw[->, ultra thick, dashed] plot[smooth]  coordinates {(3.3,1.65) (3.3, -1.85)};
\draw[<-, ultra thick] plot[smooth]  coordinates {(-0.5,-1.65) (2.2,-2.05)  (3.35,-2.1)};
\draw (3.2, 2.7) node[]{as $b$ increases $x^*$ varies slightly};
\draw (5.2, 0.3) node[align=center]{once $b$ crosses \\a threshold \\ $x^*$ jumps drastically\\(catastrophe)};
\draw (3, -3.3) node[align=center]{decreasing $b$ a little does\\ not reverse the catastrophe\\(hysteresis)};
}
\end{tikzpicture}
}
}



\end{document}
